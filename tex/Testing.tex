\subsection{Method}
To answer our hypotheses we have designed a qualitative play experience test. The procedure of the test includes first a playthrough of the game and then a semi-structured interview. The in-game choices made by the player in the playthrough is recorded as is their body language, statements or other sounds produced while playing.

The test subjects would be asked to participate in one playthrough of the game followed by a semi-structured interview to gather data on that specific play experience. Due to the mature content of the game the targeted test subjects should be age 18 or above. To look for trends across genders we aimed for gender equality in our sample size. Furthermore while recruiting test subjects we looked for subjects who had outspoken opinions about dogs, animal rights and/or human rights. A sample size of 8-10 subjects should be enough to find trends in the data. The only quantitative data collected is the in game choices of the game and these are of binary quality at best. 

In-game choices:
In the game you as player has to take several choices regarding the fate of your avatar. \commenting{Avatar or character?? it seems like it depends on the eyes that sees.. INVESTIGATE}



\subsection{Sessions}
Two testing sessions were conducted on the IT University of Copenhagen. The first on the 30th October 2017, the second on the 23rd November 2017. The test procedure followed the same guidelines for both of the tests however a few of the questions were rephrased and the implementation of the game underwent several changes. The changes included correction of spelling in the narrative, more graphical elements depicting the situation of the state of the game and better feedback in the dog fights in the game. These changes will have biased the test results and should be kept in mind when analysing the data.

A playthrough of the game took 10-15 min. depending on the speed of the participant - reading, taking decisions - and the random outcomes of the fights in the game. While playing the participants in-game choices was recorded. Some participants came to an end of the game after the second fight of the game others after the third. \commenting{Might indicate a bias}
The interview was conducted following af semi-structured questionnaire consisting of 17 predefined questions, participant demographics questions and a small observational sheet to record the actions taken by the participant in the game. Furthermore observations of the participants body language, statements and other kinds of sounds while playing the game was noted.

\subsection{Results}

In total nine subjects, four male and five female successfully participated in the test. The youngest being 24 and the oldest 31, average being 27,2 years old. Six of the participants are higher education students, two are film directors and the last participant is educated but unemployed. It is safe to conclude that the participants on average possesses intellectual judgement.

The most important choice of the game is where the player is asked to choose their destiny. Whether they will participate in dog fighting or offer sexual favours to earn money. Six participants chose dog fighting, three chose sexual favours. All but one of the participants made the choice relatively quickly between three to nine seconds. The one exemption spend a minute before choosing dog fighting.

\subsection{Discussion}
IS DOG FIGHTING THE BETTER ALTERNATIVE?
One of the things we were curious about is whether dog fighting is perceived as a better alternative to prostitution. It is no secret that in our - the designers - mind dog fighting is a better alternative. Our test data indicate that the majority of the participants felt the same way but we cannot ignore the data that resists this notion. A trend in the data reveals that the visual appearance of the man you are to prostitute yourself to has had an impact on the choice. We might also have to question whether the choice is affected by the player's cultural background.

One participant spend a minute making the decision. She was clearly emotionally affected by the game and the choice made her feel uncomfortable. Interestingly she was part of the first test session. In the implementation of the game used for the first session the man to which you can offer sexual favours is not depicted. In the second implementation of the game he is depicted as being very repulsive.  \commenting{Insert picture of man} The fact that he is not depicted in the first test session may have made the choice less unambiguous. This thought is supported by two other participants statements to the question "Why did you pick dog fighting over sexual favours?". They state that the man's uglyness played a role in their decision to pick dog fighting. A fourth participant  states that he felt the game wanted him to choose dog fighting over sexual favours. Whether the depiction of the man has had an impact on this can only be speculated but it seems like new implementation of the game pushes the player more towards picking dog fighting over sexual favours.

Two of the three participants who chose sexual favours over dog fighting stated that they chose sexual favours out of curiosity. Observations of the participants, a male and a female, indicate that they were taking the seriousness of the game lightly and were having fun with it. They seemed to have a satirical distance to the game. The male participant states that he chose the sexual favours because it was more extreme. The female states that her choice may have been biased because she was sitting at a test. All in all their data indicate that they felt that dog fighting was the moral choice but that they chose sexual favours just to see what would happen in the game. Interestingly the last participant chose sexual favours because she felt that it was the moral choice. Born and raised in Hong-Kong her cultural background is different from the rest of the participants (native Danes). Her reasoning being that no one dies in the sexual favours scenario. She is aware of, and believes that her opinion and choice is different from the other test participants. However she herself states that it is because of her cultural background.

CHARACTER / AVATAR AND DOG




