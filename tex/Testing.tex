\subsection{Method}
To answer our hypotheses we designed a play experience test in which the test participants were asked to participate in one playthrough of the game followed by an interview to gather data on that specific play experience.

\subsection{Sessions}
Two testing sessions was held on the IT University of Copenhagen the first on the 30th October 2017, the second on the 23rd November 2017. The test procedure followed the same guidelines for both of the tests however a few of the questions were rephrased and the implementation of the game underwent several changes. The changes included correction of spelling in the narrative, more graphical elements depicting the situation of the state of the game and better feedback in the dog fights in the game. These changes will without a doubt have biased the test results and should be kept in mind when analysing the data.



A playthrough of the game took about 10-15 min. depending on the speed of the participant - reading, taking decisions - and the random outcomes of the fights in the game. Some participants ended the game after the second fight of the game others after the third. \commenting{Might indicate a bias} 
The interview was conducted following af semi-structured questionnaire consisting of 17 predefined questions, participant demographics questions and a small observational sheet to record the actions taken by the participant in the game. Furthermore observations of the participants body language, statements and other kinds of sounds while playing the game was noted.

\subsection{Results}