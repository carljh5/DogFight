\subsection{Method}
To answer our hypotheses we have designed a qualitative play experience test. The procedure of the test includes a playthrough of the game followed by a semi-structured interview. The in-game choices made by the player in the playthrough is recorded as is their body language, statements or other sounds produced while playing.

The test subjects will be asked to participate in one playthrough of the game followed by a semi-structured interview to gather data on that specific play experience. Due to the mature content of the game the targeted test subjects are age 18 or above. To look for trends across genders we aim for gender equality in our sample. Furthermore while recruiting test subjects we look for subjects who has outspoken opinions about dogs, animal rights and/or human rights.

As mentioned in chapter \ref{designGoals} the prototype is hard-coded so that the player will always win at least one fight and loss the fourth fight. This makes it possible for us to determine how the testers react when their dog kills another and also when their own dog dies.\

\subsection{Sessions}
Two testing sessions were conducted on the IT University of Copenhagen. The first on the 30th October 2017, the second on the 23rd November 2017. The test procedure followed the same guidelines for both of the tests however a few of the questions were rephrased and the implementation of the game underwent several changes. The changes included correction of spelling in the narrative, more graphical elements depicting the situation of the state of the game and better feedback in the dog fights in the game. These changes will have biased the test results and should be kept in mind when analysing the data.

A playthrough of the game took 10-15 min. depending on the speed of the participant - reading, taking decisions - and the random outcomes of the fights in the game. While playing the participants in-game choices was recorded. Some participants came to an end of the game after the second fight of the game others after the third.
The interview was conducted following af semi-structured questionnaire consisting of 17 predefined questions, participant demographics questions and a small observational sheet to record the actions taken by the participant in the game. Furthermore observations of the participants body language, statements and other kinds of sounds while playing the game was recorded.

\subsection{Results} \label{Results}
In total nine subjects, four male and five female, successfully participated in the test. The youngest being 24 and the oldest 31, average being 27,2 years old. Six of the participants are higher education students, two are film directors and the last participant is educated but unemployed. It is safe to conclude that the participants on average possesses intellectual judgement. In this chapter (\ref{Results}) the data obtained from the interviews and the play test with the participants are presented. The data has been coded and analysed for relevant trends. First observations of the players decisions in-game are presented. Next, each question are presented along with a highlight of the participants responses to either the question or a general trend found in the data to that question. The participants are represented by number (1-9). Participants 2, 4, 6 and 7 are female, 1, 3, 5, 8 and 9 are male. 

\subsubsection{In-game Decisions} \label{ingamedec}

\textbf{Sharing glue.} The first in-game decision the player has to make is whether to share the sniffing glue or not. The majority of the participants shared the glue. Participant (4, 7, 9) chose not to. (Table \ref{tab:glue}).

\begin{table}[h]
\centering
\begin{tabular}{l l l}
\hline
\textbf{Shared the glue?} & Yes & No \\
\hline
Male & 4 & 1 \\
Female & 2 & 2 \\
\textbf{Total} & 6 & 3 \\
\hline
\end{tabular}
\caption{\label{tab:glue}Overview of participants in-game decisions on whether to share the sniffing glue or not}
\end{table}


\textbf{Chosen protagonist name.} The second decision the participants had to make in the game was to give the protagonist a name. Six participants (1, 2, 3, 4, 5, 7) chose to give the protagonist a, what we define as a, serious name (Table \ref{tab:name}). The serious names are names that resemble the participants real name or is a known nickname of theirs while the non-serious names are not. We speculate that the participants chose a serious name because they want to identify with the protagonist rather than distance themselves from him. While picking a non-serious name is a way to create a humorous distance between yourself and the protagonist in the game.

\begin{table}[h]
\centering
\begin{tabular}{l l l}
\hline
\textbf{Protagonist name} & Serious & Non-serious \\
\hline
Male & 3 & 2 \\
Female & 2 & 2 \\
\textbf{Total} & 5 & 4 \\
\hline
\end{tabular}
\caption{\label{tab:name}Overview of participants choice of seriousness of protagonist name}
\end{table}


\textbf{Destiny and duration of decision.} The most important decision of the game is when the player is asked to choose their destiny. Whether they will participate in dog fighting or offer sexual favours to earn money. The majority of the participants chose dog fighting, only three (1, 4, 6) chose sexual favours (Table \ref{tab:dest}). All but one of the participants made the choice relatively quickly between three to nine seconds. The one exemption, participant 2, spend a minute before choosing dog fighting.

\begin{table}[h]
\centering
\begin{tabular}{l l l}
\hline
\textbf{Chosen destiny} & Sexual favours & Dog fighting \\
\hline
Male & 1 & 4 \\
Female & 2 & 2 \\
\textbf{Total} & 3 & 6 \\
\hline
\end{tabular}
\caption{\label{tab:dest}Overview of participants in-game decisions on the destiny of the protagonist}
\end{table}


\textbf{Choice of dog.} Boxer was the favourite choice of dog among the participants. Only one participant (7) chose the Golden Retriever. Interestingly the Husky was the favourite among the male participants (1, 3, 8) while no female participants chose it (Table \ref{tab:dog}).

\begin{table}[h]
\centering
\begin{tabular}{l l l l}
\hline
\textbf{Choice of dog} & Boxer & Husky & Golden Retriever \\
\hline
Male & 2 & 3 & 0 \\
Female & 3 & 0 & 1 \\
\textbf{Total} & 5 & 3 & 1 \\
\hline
\end{tabular}
\caption{\label{tab:dog}Overview of participants choice of dog}
\end{table}


\textbf{Chosen dog name.} The chosen dog names has, like the chosen protagonist names, been divided into serious and non-serious names. Now only two participants (2, 7) chose a serious name (Table \ref{tab:dame}). 

\begin{table}[h]
\centering
\begin{tabular}{l l l}
\hline
\textbf{Dog name} & Serious & Non-serious \\
\hline
Male & 0 & 5 \\
Female & 2 & 2 \\
\textbf{Total} & 2 & 7 \\
\hline
\end{tabular}
\caption{\label{tab:dame}Overview of participants choice of seriousness of dog name}
\end{table}


\textbf{Leashing dog.} Only two participants (6, 8) chose to leash their dog (Table \ref{tab:leas})

\begin{table}[h]
\centering
\begin{tabular}{l l l}
\hline
\textbf{Leashed the dog?} & Yes & No \\
\hline
Male & 1 & 4 \\
Female & 1 & 3 \\
\textbf{Total} & 2 & 7 \\
\hline
\end{tabular}
\caption{\label{tab:leas}Overview of participants in-game decisions on whether to leash their dog or not}
\end{table}


\textbf{Weekly options.} Unfortunately the decision of the weekly option (see chapter \ref{weekly}) was not recorded for the participants (1, 2) in the first test session. However in the second test session all of the participants chose to train their dogs.


\subsubsection{Interview} \label{interview}
\textbf{Q1.} A trend in the respones to the question "How would you characterise the experience?" is the participants description of the emotional impact the play experience had on them, see Table \ref{tab:emo}. The majority of the participants felt a negative emotional impact. Only one participant (6) does not express any emotional relation to the play experience. Her response implies that there was too much text in the game for her to be engaged.

\begin{table}[h]
\centering
\begin{tabular}{l l l l}
\hline
\textbf{Emotional impact from}\\
\textbf{play experience?} & Negative & Positive & Other \\
\hline
Male & 3 & 2 & 0 \\
Female & 3 & 0 & 1 \\
\textbf{Total} & 6 & 2 & 1 \\
\hline
\end{tabular}
\caption{\label{tab:emo}Emotional impact described in the responses to the question "How would you characterise the experience?"}
\end{table}

\textbf{Q2.} The responses to the question "How did you like your character?" are diverse and raises an interesting question; what is the participants perceived relation to the protagonist in the game? A trend in the data shows that the majority of the participants feels sympathy for the protagonist and/or his situation in the game (Table \ref{tab:symp}). Interestingly only two of the female participants (2, 7) express sympathy for the protagonist which raises the question whether the gender of the protagonist (male) has an influence on the felt sympathy towards him. However studying the responses it becomes clear that the participants has different interpretations of their relationship as players towards the protagonist. Participant (4) does not really answer the question but instead respond that it is just her self making decisions, hence she did not have any emotional connection to the protagonist. Participant (3) on the other hand states that the sparse description of the protagonist enabled him to project his self onto the protagonist. A third participant (8) states that he think the kid is sweet and that he want to follow him around, implying that he does not control the protagonist but merely the circumstances in which the protagonist is placed.

\begin{table}[h]
\centering
\begin{tabular}{l l l l}
\hline
\textbf{Sympathy for the}\\
\textbf{protagonist?} & Yes & No & Other \\
\hline
Male & 5 & 0 & 0 \\
Female & 2 & 2 & 0 \\
\textbf{Total} & 7 & 2 & 0 \\
\hline
\end{tabular}
\caption{\label{tab:symp}Expression of sympathy towards the protagonist in the responses to the question "How did you like your character?"}
\end{table}

\commenting{Man kan godt tage en moralsk beslutning selvom begge valg er uetiske}
\textbf{Q3.}  As can be seen in Table \ref{tab:choice}, the majority of the participants expressed that the moral option was to choose dog fighting over sexual favours in the game. Only one participant (4) felt sexual favours was the right option, but elaborates that her ethics might be different than the rest of the testers because of her cultural background. Participant 4 is born and raised in Hong-Kong as opposed to the rest of the testers who's ethnicity is Danish. Participant 8 chose dog fighting because he felt that the game wanted him to do so and participant 9 chose dog fighting because the man to which you can offer sexual favours was too repulsive. Participant 7 also mentions the man's appearance as one of the deciding factors for her choice.

\begin{table}[h]
\centering
\begin{tabular}{l l l l}
\hline
\textbf{Sexual favours or dog fighting,}\\
\textbf{which is the moral option?} & Sexual favours & Dog fighting & Other \\
\hline
Male & 0 & 3 & 2 \\
Female & 1 & 3 & 0 \\
\textbf{Total} & 1 & 6 & 2 \\
\hline
\end{tabular}
\caption{\label{tab:choice}Expression of moral in responses to the question "Why did you pick sexual favours or dog fighting over the other?"}
\end{table}

\textbf{Q4.} Judging from the responses to the question "Did you feel uncomfortable making the choice?" most of the participants were comfortable making their decision(Table \ref{tab:unco}). Unfortunately participant 2's response to the question is lost, but the observations recorded at her play session indicate that it was a very difficult choice for her. Apart from her body language signifying her being uncomfortable, she spend approximately one minute deciding which option to choose. In contrary the rest of the participants spend between 3-9 seconds. 

\begin{table}[h]
\centering
\begin{tabular}{l l l l}
\hline
\textbf{Did you feel uncomfortable}\\
\textbf{making the choice?} & Yes & No & Other \\
\hline
Male & 1 & 4 & 0 \\
Female & 2 & 2 & 0 \\
\textbf{Total} & 3 & 6 & 0 \\
\hline
\end{tabular}
\caption{\label{tab:unco}Responses to the question "Did you feel uncomfortable making the choice?"}
\end{table}


\textbf{Q5.} The participants choice of dog described in their responses to the question "Why did you pick the dog you picked?" are coded as being either a choice based on feelings or rationale (Table \ref{tab:rati}). Six participants (2, 3, 4, 5, 6, 7, 9) made a rational decision on their choice of dog. They base their decision on which dog they felt was best capable of fighting. In contrary participant 1 and 8 based their choice of dog on feelings. Participant 1 chose the husky because he thought it was cool but speculate that the boxer might have been more capable of fighting. Participant 8 did not choose the boxer because he was too emotionally attached to it and he wouldn't want to see it suffer. His response raises the question whether the other participants seemingly rationale choices are made on the basis of choosing the dog that would suffer the least. If that is the case then the participants choices are based on feelings in the sense that they choose the dog that will have the least emotional impact on them.

\begin{table}[h]
\centering
\begin{tabular}{l l l l}
\hline
\textbf{Rational or emotional}\\
\textbf{choice of dog?} & Rational & Emotional & Other \\
\hline
Male & 3 & 2 & 0 \\
Female & 4 & 0 & 0 \\
\textbf{Total} & 7 & 2 & 0 \\
\hline
\end{tabular}
\caption{\label{tab:rati}Expression of rational or emotional based choice in responses to the question "Why did you pick the dog you picked?"}
\end{table}


\textbf{Q6.} As can be seen in Table \ref{tab:like}, the participants liked their dogs. Only one participant (5) express that he did not feel a relation to the dog. However he also expresses that he chose a non-serious name to avoid getting too close to the dog. Interestingly two other participants (3, 9) also express a similar rationale for choosing a non-serious name for their dog. 

\begin{table}[h]
\centering
\begin{tabular}{l l l l}
\hline
\textbf{Did you like}\\
\textbf{your dog?} & Yes & No & Other \\
\hline
Male & 4 & 0 & 1 \\
Female & 4 & 0 & 0 \\
\textbf{Total} & 8 & 0 & 1 \\
\hline
\end{tabular}
\caption{\label{tab:like}Responses to the question "Did you like your dog?"}
\end{table}


\textbf{Q7.} The responses to the question "What would happen if you leashed your dog" can be divided into three categories. Leashing the dog would affect the relationship to the dog, the control over the dog or a believe that the action of leashing was a choice of right and wrong in terms of progressing in the game. The last notion has been coded as "other", see Table \ref{tab:leash}. Only two participants (6, 8) chose to leash the dog however both of them expressed that they thought it was the correct option, implying that they were afraid of the consequences in terms of progression in the game if they did not leash it. Participant 1, 2, 3, 7 and 9 express that they believe their relationship with the dog would be improved by not leashing it. Participant 4 and 5 both express that leashing it would have made the dog easier to control, but chose to not leash it. Which raises the question; what was their consideration behind not leashing it? Do they, in line with the majority of the participants, think that not leashing the dog can improve their relationship or do they too feel that the game proposes a right / wrong solution,

\begin{table}[h]
\centering
\begin{tabular}{l l l l}
\hline
\textbf{Believed consequence of}\\
\textbf{leashing dog?} & Relationship & Control & Other \\
\hline
Male & 3 & 1 & 1 \\
Female & 2 & 1 & 1 \\
\textbf{Total} & 5 & 2 & 2 \\
\hline
\end{tabular}
\caption{\label{tab:leash}Expression of believed consequences of leashing the dog in responses to the question "What would happen if you leashed your dog?"}
\end{table}


\textbf{Q8.} Only two participants (1, 2) felt responsible for the dogs their dog killed. It is worth noting that these were the two participants who participated in the first test session which may indicate a bias. Participant 4 and 7 are vague in their responses. Participant 7 states that she felt a bit responsible but that she would rather have that the other dog died than her own dog died. (Table \ref{tab:resp})

\begin{table}[h]
\centering
\begin{tabular}{l l l l}
\hline
\textbf{Did you feel responsible for}\\
\textbf{the dogs your dog killed?} & Yes & No & Other \\
\hline
Male & 1 & 4 & 0 \\
Female & 1 & 1 & 2 \\
\textbf{Total} & 2 & 5 & 2 \\
\hline
\end{tabular}
\caption{\label{tab:resp}Responses to the question "Did you feel responsible for the dogs your dog killed?"}
\end{table}


\textbf{Q9.} In general the participants express that they feel sadness when their dog died. As can be seen in Table \ref{tab:sad} three participants did not have a clear cut answer to the question. Participant 4 elaborates that she had prepared herself for it the moment she caught the dog. She knew that death was inevitable, implying that she was emotionally prepared for it to happen. Her response implies that she had to emotionally detach herself from the dog to not feel sad which in turn implies that she would have felt sad if she hadn't predicted its destiny. Participant 6 felt sad when the dog died not because of any emotional attachment to the dog but because she lost the game. Lastly participant 9 express that he did not feel sad, instead he felt anger towards the dog that killed his dog. Where the anger stems from is unclear and we can only speculate that either he was in fact emotionally attached to his dog or that he, in line with participant 6, was sad that he lost the game.

\begin{table}[h]
\centering
\begin{tabular}{l l l l}
\hline
\textbf{Did you feel sad when}\\
\textbf{your dog died?} & Yes & No & Other \\
\hline
Male & 3 & 1 & 1 \\
Female & 2 & 0 & 2 \\
\textbf{Total} & 5 & 1 & 3 \\
\hline
\end{tabular}
\caption{\label{tab:sad}Responses to the question "Did you feel sad when your dog died?"}
\end{table}


\textbf{Q10.} The majority of the participants felt responsible for the death of their dog (Table \ref{tab:death}). Only one participant (5) states that because he tried to run from the battle it was not his responsibility. Participant 4 states that she decided that she did not want to feel anything. Lastly participant 6 express that because she felt she was cheated she did not feel a great degree of responsibility.

\begin{table}[h]
\centering
\begin{tabular}{l l l l}
\hline
\textbf{Did you feel responsible for}\\
\textbf{the death of your dog?} & Yes & No & Other \\
\hline
Male & 4 & 1 & 0 \\
Female & 2 & 0 & 2 \\
\textbf{Total} & 6 & 1 & 2 \\
\hline
\end{tabular}
\caption{\label{tab:death}Responses to the question "Did you feel responsible for the death of your dog?"}
\end{table}


\textbf{Q11/Q12.} To understand the participants experience of perceived agency on the fights of the game they were asked two questions. "How does the battle system work?" and "Which options were most effective?". The participants responses to the questions has been coded as either the participant believed that hey had agency in the fights or they did not. The responses are illustrated in Table \ref{tab:agen}. Only one participant's (2) statement is unclear. She believes she can use items in the fights later in the game. However she also express that it seemed like her options had no effect. The majority of the participants believes they have agency in the fights though some of them are confused about the system. Only one participant (9) believed that he had no agency. He states that the realisation of his lack of agency came in the beginning of the second fight he was in. He states that it changed his perception of his role in the game, that he felt like he was becoming a spectator and that this notion made the message of the game stronger. His definition of the game's message is vague. He thinks the game is trying to convey that the protagonist is in a catch-22, which in turn creates awareness about poverty and what you have to do  to survive.

\begin{table}[h]
\centering
\begin{tabular}{l l l l}
\hline
\textbf{Perceived agency}\\
\textbf{in the fights?} & Agency & No agency & Other \\
\hline
Male & 4 & 1 & 0 \\
Female & 3 & 0 & 1 \\
\textbf{Total} & 7 & 1 & 1 \\
\hline
\end{tabular}
\caption{\label{tab:agen}Expression of perceived agency in the responses to the questions "How does the battle system work?" and "Which options were most effective?"}
\end{table}


\textbf{Q13.} To the question "Do you think it was an unethical game?", the majority of the participants answered no. Participant 6 answered that it did not matter as the game is satire (Table \ref{tab:ethi}). Only participant 5 and 8 answered yes. Participant 5 elaborates that he believes the Mexicans will oppose the game, but that in a sense the game is not more unethical than other games. He believes that human fights is more unethical, implying that the while unethical the game can exist under the socially accepted threshold of unethically in games. Participant 8 elaborates that the humor in the game makes it very unethical, but that its existence is still justified. Participants 1, 2, 3, 4 & 7 elaborates that the game makes you are aware and puts you in the horrible situation of the game, implying that your actions are justified by the situation you are in. Participant 4 and 7's only real complaint was that they wanted a disclaimer for the blood on the dogs.

\begin{table}[h]
\centering
\begin{tabular}{l l l l}
\hline
\textbf{Do you think it was}\\
\textbf{an unethical game?} & Yes & No & Other \\
\hline
Male & 2 & 3 & 0 \\
Female & 0 & 3 & 1 \\
\textbf{Total} & 2 & 6 & 1 \\
\hline
\end{tabular}
\caption{\label{tab:ethi}Responses to the question "Do you think it was an unethical game?"}
\end{table}


\textbf{Q14.} When asked if the game was moralising the majority of the participants answered no. Participant 4 answered yes and participant 7 was not sure (Table \ref{tab:mora}). Participants 1, 2, 3 and 5 elaborates that the game itself does not tell you what is right or wrong. Participant 4 states that the game is trying to hint something. She is not defining what the game is hinting but says the game is made to make you feel bad by dismembering and putting blood on cute dogs. Participant 7 also cannot define what the moral of the game is but that she feels it is tough that the game forces her to downprioritise her dog.

\begin{table}[h]
\centering
\begin{tabular}{l l l l}
\hline
\textbf{Is the game}\\
\textbf{moralizing?} & Yes & No & Other \\
\hline
Male & 0 & 5 & 0 \\
Female & 1 & 2 & 1 \\
\textbf{Total} & 1 & 7 & 1 \\
\hline
\end{tabular}
\caption{\label{tab:mora}Responses to the question "Is the game moralising?"}
\end{table}


\textbf{Q15.} To the question "What do you think about dog fighting?" all of the participants expressed disgust of a varying degree towards it. 


\textbf{Q16.} As can be seen in Table \ref{tab:thin} six participants express that the game has made them think more about dogfighting. However four of them express that they only do so temporarily while playing the game or only until short after having stopped playing. One participant (4) express that she does not think about dogfighting because there is nothing to do about it. Another participant (3) express that the game has made him think about the setting than the actual dogfights.

\begin{table}[h]
\centering
\begin{tabular}{l l l l}
\hline
\textbf{Has this game made you think}\\
\textbf{more about dogfighting?} & Yes & No & Other \\
\hline
Male & 4 & 0 & 1 \\
Female & 2 & 1 & 1 \\
\textbf{Total} & 6 & 1 & 2 \\
\hline
\end{tabular}
\caption{\label{tab:thin}Responses to the question "Has this game made you think more about dog fighting?"}
\end{table}


\textbf{Q17.} All of the participants express, to a varying degree, that they believe dogfighters love their dogs. Participant 8 furthermore express that he believe that the dogfight spectators do not love dogs. A belief participant 2 and 4 share.