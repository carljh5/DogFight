\subsection{Method}
To answer our hypotheses we have designed a qualitative play experience test. The procedure of the test includes first a playthrough of the game and then a semi-structured interview. The in-game choices made by the player in the playthrough is recorded as is their body language, statements or other sounds produced while playing.

The test subjects would be asked to participate in one playthrough of the game followed by a semi-structured interview to gather data on that specific play experience. Due to the mature content of the game the targeted test subjects should be age 18 or above. To look for trends across genders we aimed for gender equality in our sample size. Furthermore while recruiting test subjects we looked for subjects who had outspoken opinions about dogs, animal rights and/or human rights. A sample size of 8-10 subjects should be enough to find trends in the data. The only quantitative data collected is the in game choices of the game and these are of binary quality at best. 

In-game choices:
In the game you as player has to take several choices regarding the fate of your avatar. \commenting{Avatar or character?? it seems like it depends on the eyes that sees.. INVESTIGATE}

\subsection{Sessions}
Two testing sessions were conducted on the IT University of Copenhagen. The first on the 30th October 2017, the second on the 23rd November 2017. The test procedure followed the same guidelines for both of the tests however a few of the questions were rephrased and the implementation of the game underwent several changes. The changes included correction of spelling in the narrative, more graphical elements depicting the situation of the state of the game and better feedback in the dog fights in the game. These changes will have biased the test results and should be kept in mind when analysing the data.

A playthrough of the game took 10-15 min. depending on the speed of the participant - reading, taking decisions - and the random outcomes of the fights in the game. While playing the participants in-game choices was recorded. Some participants came to an end of the game after the second fight of the game others after the third. \commenting{Might indicate a bias}
The interview was conducted following af semi-structured questionnaire consisting of 17 predefined questions, participant demographics questions and a small observational sheet to record the actions taken by the participant in the game. Furthermore observations of the participants body language, statements and other kinds of sounds while playing the game was noted.

\subsection{Results}

In total nine subjects, four male and five female successfully participated in the test. The youngest being 24 and the oldest 31, average being 27,2 years old. Six of the participants are higher education students, two are film directors and the last participant is educated but unemployed. It is safe to conclude that the participants on average possesses intellectual judgement.

The most important choice of the game is where the player is asked to choose their destiny. Whether they will participate in dog fighting or offer sexual favours to earn money. Six participants chose dog fighting, three chose sexual favours. All but one of the participants made the choice relatively quickly between three to nine seconds. The one exemption spend a minute before choosing dog fighting.

\subsection{Discussion}
In the following paragraph test subjects will be referred to by number 1-9. 

IS DOG FIGHTING THE BETTER ALTERNATIVE?
One of the things we were curious about is whether dog fighting is perceived as a better alternative to prostitution. It is no secret that in our - the designers - mind dog fighting is a better alternative. Our test data indicate that the majority of the participants felt the same way but we cannot ignore the data that resists this notion. A trend in the data reveals that the visual appearance of the man you are to prostitute yourself to has had an impact on the choice. We might also have to question whether the choice is affected by the player's cultural background.

One participant(2) spend a minute making the decision. She was clearly emotionally affected by the game and the choice made her feel uncomfortable. Interestingly she was part of the first test session. In the implementation of the game used for the first session the man to which you can offer sexual favours is not depicted. In the second implementation of the game he is depicted as being very repulsive.  \commenting{Insert picture of man} The fact that he is not depicted in the first test session may have made the choice less unambiguous. This thought is supported by two other participants(7, 9) statements to the question "Why did you pick dog fighting over sexual favours?". They state that the man's uglyness played a role in their decision to pick dog fighting. A fourth participant(8) states that he felt the game wanted him to choose dog fighting over sexual favours. Whether the depiction of the man has had an impact on this can only be speculated but it seems like new implementation of the game pushes the player more towards picking dog fighting over sexual favours.

Two(1, 6) of the three participants who chose sexual favours over dog fighting stated that they chose sexual favours out of curiosity. Observations of the participants, a male and a female, indicate that they were taking the seriousness of the game lightly and were having fun with it. They seemed to have a satirical distance to the game. The male participant(1) states that he chose the sexual favours because it was more extreme. The female(2) states that her choice may have been biased because she was sitting at a test. All in all their data indicate that they felt that dog fighting was the moral choice but that they chose sexual favours just to see what would happen in the game. Interestingly the last participant(4) chose sexual favours because she felt that it was the moral choice. Born and raised in Hong-Kong her cultural background is different from the rest of the participants (native Danes). Her reasoning being that no one dies in the sexual favours scenario. She is aware of, and believes that her opinion and choice is different from the other test participants. However she herself states that it is because of her cultural background.

CHARACTER / AVATAR AND DOG
A central question regarding the game design is how emotionally invested the players are in the game. Our goal was to create a play experience that had en emotional impact on the player that goes beyond pure entertainment. We believed that this could be achieved by first creating a gloomy narrative set in a - close to - real environment and by creating affordances that makes the player experience the story as being the protagonist rather than observing.  The participants identification with the depicted protagonist of the game and the participants perceived roles in the game indicates that the participants played the game in different ways. We cannot quantify the participants emotional investment but there are trends in the data that indicate how the players felt while playing the game. The data indicates that the players have different perceptions of the protagonist as an avatar. To some he is a rounded character whos fate the players decide, to others he is a projection of player's self in his reality. To avoid a confusion of the character versus the avatar we will refer to the protagonist as the 'protagonist' in the following paragraph.

The second decision the participants had to make in the game was to give the protagonist a name. Six participants(1, 2, 3, 4, 5, 7) chose to give the protagonist a what we define as a serious name. The serious names are names that resemble the participants real name or is a known nickname of theirs while the non-serious names are not. We speculate that the participants chose a serious name because they want to identify with the protagonist rather than distance themselves from him. While picking a non-serious name is a way to create a humorous distance between yourself and the protagonist in the game. This speculation is partly supported by the data. One of the participants(6) who chose a non-serious name was experiencing the whole game as a satire, she clearly had a humorous distance to the entire play experience. Another participant(9) who chose a non-serious name states that he did not identify with the protagonist and that it was a conscious decision to give the protagonist a non-serious name. However data from three of the participants(1, 4, 7) who chose a serious name indicates that they did not identify with the protagonist. Hence the name to protagonist relationship hypothesis does not seem to apply. Instead choosing a serious name may be an expression of how willing you are to invest emotions in the game.

Interestingly the serious name pattern changes when the participants are asked to give their dog a name. Now only two participants(2, 7) chose to give their dog a serious name. This shift might indicate either that the game has made the participants perceive their role in the game differently or that the participants are trying consciously to distance themself from the either the game or the dog in the game. Participant 5 and 9 states that they chose a funny or distancing name to avoid getting too close to the dog. The data of participant 4 indicate the same trend but is less clear cut. Participants 1, 3 and 8's data indicate that they did not yet feel attached to the dog the moment they had to give it a name. Participant 6 was as explained earlier having a satirical distance to the entire game experience.
Interestingly only two participants(6, 8) chose to leash the dog however both of them expressed that they thought it was the correct option, implying that they were afraid of the consequences if they did not leash it. Participant 1, 2, 3, 7 and 9 express that they believe their relationship with the dog would be improved by not leashing it. Participant 4 and 5 both express that leashing it would have made the dog easier to control, but chose to not leash it. The findings indicate that the name can be perceived as an indication of emotional engagement in the game. Moreover it seems that our implementation did a sub par job at creating the necessary emotional bonding to the dog prior to the naming of it.

TO-DO:...
Take the previous written stuff and see if it fits as answers to the three hypotheses. Find more trends in the data. Structure the whole testing paragraph in a better way. At least make subsection titles better.








