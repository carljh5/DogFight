From the test results in section \ref{Results}, we are able to conclude that our method for using sensitive subject matter worked, and hence the possibility of creating a game about dog fighting is artistically feasible. In this section we will discuss under which conditions the prototype would be best evolved into an actual game.\

\subsection{Potential Game Mechanics}
The setting of a street kid in Mexico City seemed to work very well, as most testers deemed it to be both authentic and serve as a valid cause for the player character to start a career in dog fighting. The delivery of the narrative was not especially subtle or elegant, and especially Tester 6, found the text-based story-telling to be slow and verbose. For a full game the narrative elements would have to be shortened a lot and told more cinematically, possibly by creating a small animated intro movie. \

In the prototype the selection of the \textit{Sexual fovours}-option would make the game restart, as a way to mechanically show that the street kid has to be proactive in order to create a change in his life. Three testers picked this option, two of them to test the boundaries of the game, and one of them (Tester 4) because she thought it was the most ethical choice. While interesting artistically it is not very good game design to have, what is in its essence, an auto-loss button. Our options are then to either implement the child prostitution as actual game content or have the choice made for the player, possibly by the player character.\

It would be interesting to let the player suffer the consequenses of their choises by incorporating child prostitution in the game, possibly even gamifying the actual sexual act. This would both increase the game and scope and make it more of a street kid simulation game, which would take away the focus of the game from dog fighting. We therefore find the best solution for the game to be that the player character makes the choice of dog fighting for the player, while still explaining the situation and reasons for doing so. Since all testers, other than Tester 6, found this to be a realistic choice and situation, we believe that players would still find it reasonable, even if the player character makes the choice for them.\

None of our testers understood, that there was no agency in the battles, while they did comment on a lack of understanding of how the selectable options worked. It should therefore be possible to make combat actually have agency, without the game becoming too unethical. Still, we believe the gameplay would be better served without any agency in the battle, where the player knows that there is not any agency. Instead the player could have the possibility of having the player character shout different commands to the dog in the battles. This could be similar to the shouts from the actions we have currently, as mentioned in \ref{Design}. This would more legitimately position the player in the same position as the kid, hoping that the shouts might change something, instead of expecting it.\

Instead of combat agency the gameplay should focus on the places where the player actually is able to make an impact; training, breeding, catching dogs, economy management and random events. Here the player should be able to strategize and plan, so the important part would be which dogs they pit against which, catching or breeding good fighting dogs, and chosing when the stakes are high enough to throw your prime bitch into the ring. This should allow for plenty of design space for strategy and actual gameplay, while still keeping the realism and brutality of the fights. \

While we do believe that some amount of humour makes a game, that could be described as social realism, more bearable, the humour of the game could also turn off some potential players, like Tester 8, who found the game to become unethical specifically because of the humour involved. So less humour in the game could make it more realistic, specifically removing the Pokémon references, which is already a logical move, if we are no longer trying to create the expectancy of agency in the fights.\

\subsection{Audience and Platform}

As mentioned in \ref{Results}, one of our testers even suggested collaborating with NGO's, to create awareness around the conditions of street kids and illegal dog fighting. \

For the platform we believe both art exibitions, schools and retail release could all be possible. Interestingly, we believe that the latter would need more attention to the ethical considerations, possibly even necessitating the downgrade of some of the systems to a less game-like structure. On the other hand developing the game as an art project, meant to be shown at a museum or art exibition, would just by the context establish the game as a piece of art, and allow, possibly even expect, more extreme forms of expression, in regards to humour and affordances.
A game where the intended purpose is either educational or artistic, would also be easier to find funding for. 
Hence, we find that an initial release for either schools or museums, with the possibility of a later retail release on Steam\footnote{\url{http://store.steampowered.com/}} would be ideal.\

One problem with collaborating with schools or NGO's, is that the players might find the game more moralizing, even if would not in another context, simply because of these associations. No matter which platform is chosen, it is important that the features reflect the audience and agenda, especially since people are very influenced by this subject matter, as we witnessed in testing.\