In this section we will discuss under which conditions the prototype would be best evolved into an actual game.

\subsection{Game Mechanics}

More of a real game could be done

Setting seemed to work very well. \commenting{ethics book reference} Could be shortened or shown more cinematically.

the selection of sex or dog fight could be done by the character, as in many games (ref gta), as to not make a street kid simulator.

As mentioned in \ref{testResults} none of our testers understood, that there was no agency in the battles, while they did comment on a lack of understanding of how the selectable options worked. It should therefore be possible to make combat actually have agency, without the game becoming too unethical. Still, we believe the gameplay would be better served without any agency in the battle, and the ideal solution would then be one, where the player realized that there is not any agency.

shouts instead of pokemon interface - not cheating, but focus on and expand where the player actually has agency.


Focus on where the actual gameplay is. Training, breeding catching and random events.




Less humour and pokémon references could make it a more ethical game.

\subsection{Audience and Platform}

As mentioned in \ref{testResults}, one of our testers even suggested collaborating with NGO's, to create awareness around the conditions of street kids and illegal dog fighting. 

Both art exibition, schools and retail/steam, could be possible. Interestingly, we believe that the latter would need more attention to the ethical considerations, possibly even downgrading the 
A game where the intended purpose is either educational or artistic, would also be easier to find funding for. 
Hence, we find that an initial release for either schools or museums, with the possibility of a latter release on Steam \commenting{insert reference} would be ideal.

One problem with collaborating with schools or NGO's, is that the players might find the game more moralizing, even if would not in another context, simply because of these associations.

 Important that the features reflect the audience and agenda.