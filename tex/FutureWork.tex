In this section we will discuss under which conditions the prototype would be best evolved into an actual game.

\subsection{Game Mechanics}

More of a real game could be done\

Setting seemed to work very well. \commenting{ethics book reference} Could be shortened or shown more cinematically.\

the selection of sex or dog fight could be done by the character, as in many games (ref gta), as to not make a street kid simulator.\

As mentioned in \ref{testResults} none of our testers understood, that there was no agency in the battles, while they did comment on a lack of understanding of how the selectable options worked. It should therefore be possible to make combat actually have agency, without the game becoming too unethical. Still, we believe the gameplay would be better served without any agency in the battle, where the player knows that there is not any agency. Instead the player could have the possibility of having the player character shout different commands to the dog in the battles. This could be similar to the shouts from the actions we have currently, as mentioned in \ref{Design}. This would more legitimately position the player in the same position as the kid, hoping that the shouts might change something, instead of expecting it.\

Instead of combat agency the gameplay should focus on the places where the player actually is able to make an impact; training, breeding, catching dogs, economy management and random events. Here the player should be able to strategize and plan, so the important part would be which dogs they pit against which, catching or breeding good fighting dogs, and chosing when the stakes are high enough to throw your prime bitch into the ring. This should allow for plenty of design space for strategy and actual gameplay, while still keeping the realism and brutality of the fights. \

Less humour and pokémon references could make it a more ethical game.

\subsection{Audience and Platform}

As mentioned in \ref{testResults}, one of our testers even suggested collaborating with NGO's, to create awareness around the conditions of street kids and illegal dog fighting. \

Both art exibition, schools and retail/steam, could be possible. Interestingly, we believe that the latter would need more attention to the ethical considerations, possibly even downgrading the 
A game where the intended purpose is either educational or artistic, would also be easier to find funding for. 
Hence, we find that an initial release for either schools or museums, with the possibility of a latter release on Steam \commenting{insert reference} would be ideal.\

One problem with collaborating with schools or NGO's, is that the players might find the game more moralizing, even if would not in another context, simply because of these associations.

 Important that the features reflect the audience and agenda.