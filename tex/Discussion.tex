The test data is compiled from two test sessions at ITU. As already mentioned in chapter \ref(sessions) differences in the interview questions and implementation of the game has probably biased the test results. Before we discuss our test results in the perspective of our initial research questions \ref{ProbStat} we will present the bias of the test results and argue for its validity. \\

Despite the differences between the two test sessions and the bias it may have created we believe that the data is comparable across sessions as long as we look for and document the outlying differences between the data of the two test sessions. In fact comparing the sessions can help us get indications of what impact the differences in the implementation of the game has had. The differences are hard to point out but two interesting things stands out. The first is that only the two, session 1, participants (1, 2) felt responsible for the dogs their dog killed. The second being that the depiction of the man to which you offer sexual favours apparently has had an effect on the participants choice of destiny. \\

As mentioned in the paragraph of "Q4" in chapter \ref{interview}, participant 1 spend one minute deciding her destiny (sexual favours versus dog fighting). Participant 2 on the other hand chose sexual favours in a matter of seconds. The participants (1, 2) behaviour is unlike the behaviour of the majority of the test participants in session 2. It seems like the session 1 participants were more in doubt of which of the options were most ethically sound. The reason behind this may be found in the fact that the man to which you could offer sexual favours only were depicted in the second implementation of the game \citep[see]{gringo}. Participant 7 and 9 implied that their choice of destiny partly depended on the looks of the man, that he was too repulsive. Participant 8 state that he felt the game wanted him to choose dog fight. We can only speculate what made him feel like that but the depiction of the man as being very repulsive may be the answer. \\

Why only the session 1 participants felt responsible for the dogs their dog killed may stem back to the problem mentioned above. If the choice of destiny was harder to justify for the participants (1, 2) they may have carried that sense of doubt with them into the fights questioning whether dog fighting was the moral choice as their dog kills the other dog. \\

The results in chapter \ref{interview} was presented divided by gender. However there does not seem to be a lot of interesting gender differences in the data. The one that stands out the most is the choice of dogs. The majority of the participants chose the boxer on the basis that it was best fit for dog fighting. However among the male participants the husky was the favourite and no females picked the husky. It seems like the male participants were more willing to run the risk of choosing a more 'cool looking' dog over the safe bet of the boxer. This points us in the direction that the fear of seeing the dog get hurt or killed in a fight were a deciding factor for the female participants.  \\


\begin{figure} 
	\centering
    \includegraphics[width=0.5\textwidth]{gringo.png}
    \caption{The man to which you can offer sexual favours in the game}
    \label{fig:gringo}
\end{figure}


\textbf{RQ1: Can we create unethical affordances, namely dog fighting, in a game without making it an unethical game?}\

The participants responses to Q13 and Q14 indicate that our game is neither perceived as being unethical nor moralising. The trend in the data shows us that the participants bought the premise of the game and felt that the situation you - the protagonist - is placed in justifies your otherwise unethical actions. Unfortunately their responses does not answer our research question in the general sense the question is asked. We know how test participants react to the game after having played it, but we do not have any data on how the common consumer will react to a game like this once it surfaces in public. The game contains enough sensitive subjects to create a solid foundation for public outcry. Even before its release in 2009 \textit{Resident Evil 5} \citep[RE5]{game:re} med massive criticism for its portrayal and dealing with African culture and Africans as zombies \citep{harrer2015black}. Another example from 2015 is when Serious Games had to remove a mini-game, popularly known as "Slave Tetris", in their educational serious game \textit{Playing History 2: Slave Trade}  due to a public outcry \citep{kotaku}. Common to both games is that people defending them claim that the games are subject to misinterpretation by people who have not played them \citep{harrer2015black, kotaku, mtv}. Without going further into the arguments for and against the games it seems like games containing controversial subject matter are subject to superficial interpretations of the games purposes. A hypothesis supported by Deterding's \citeyear{deterding2016mechanic} work showing that the framing of the game plays an important part in the reception of the game.

\blockquote{...Playing History 2 travelled public discourse in the form of a single screenshot of the ?Slave Tetris? level, designed in a child-friendly cartoon look, cutting away the internal critique of the modeled proceedings later in the game. By framing itself as a game for children, it activated a children?s entertainment game frame, which clashed with the serious subject in the audience?s perception.}\citep{deterding2016mechanic}\

Whether our framing of \textit{Rat Kid Dog Fight!} publicly justifies its existence we cannot know. One participant (5) expressed that he believed the Mexicans would oppose the game. The implementation of the game is very much a prototype and its fair to say that the depiction of Mexico City's slum culture is based on loose evidence and gut feelings. The above example of games that fail to frame their purpose should stand as a reminder that we have to carefully analyse and design for the right framing of the game in a future implementation.


\textbf{RQ2:  Will the player be able to feel empathy for the dogs, while being reliant on them fighting?}
The unanimous responses to Q17 ("Do dog fighters love their dogs?") illustrates well that the players had or could imagine having an emotional relation to their fighting dog. However it is hard to find a clear cut answer to whether our test participants felt empathy for their dogs in the game. To try and deduct an answer from the test results we have to broaden our investigation and look for trends that reveal the participants emotional engagement with the game starting with the participants relation to the protagonist of the game. In the paragraph of "Chosen protagonist name" in chapter \ref{ingamedec} we raise the question whether the choice of name for the protagonist is a reflection of the participants emotional engagement with the game. \

One of the participants (6) who chose a non-serious name was experiencing the whole game as a satire, she clearly had a humorous distance to the entire play experience. Another participant (9) who chose a non-serious name states that he did not identify with the protagonist and that it was a conscious decision to give the protagonist a non-serious name. However data from three of the participants (1, 4, 7) who chose a serious name indicates that they did not identify with the protagonist. The data does not provide a conclusive answer but when correlated with the data of the questions involving the participants dogs it seems like the choice of name for both dog and protagonist is a reflection of how willing the participant is to let herself be emotionally engaged in the game. 

\In contradiction to the chosen protagonist name only two participants chose to give their dog a serious name. As described in the paragraph of "Q6" in chapter \ref{interview} three participants expressed that they chose a non-serious name for their dog to avoid getting too close to it emotionally. This points us in the direction that the participants use the naming of both the dog and the protagonist to adjust their emotional engagement to a level they are comfortable with. We interpret the shift in serious names from 5 (protagonist) to 2 (dog) as a sign of raised emotional engagement over the course of the game. The players choose to give the dogs non-serious names not because it is an expression of their relation to the dog itself but a way of adjusting their emotional engagement so they are less vulnerable for whatever they would meet later in the game. This correlates well with the data of both Q6 and Q7 in chapter \ref{interview}. 

\The participants liked their dogs and they chose not to leash them to avoid damaging their relationship to them. Furthermore the majority of the participants expressed sadness upon the dogs death and a sense of responsibility towards the dog. The discussion presented in this paragraph show that the players try to create a distance to the dogs and the game via non-serious naming of their dog. Upon naming the dog the participants were aware that the dog was going to be used in for dog fighting. Even so they still made decisions that signifies emotional bonding to the dog. Hence we believe the majority of the participants felt empathy for their dog. 



\textbf{RQ3: How does the player feel, when they lose their dogs, while lacking the agency to prevent it?}
This research question can hardly be answered based on the data acquired. Only one participant (9) realised that he did not have any agency in the fights. He expressed his experience as changing from being an actor to becoming a spectator to the fights this experience he felt conveyed the message of the game in a stronger sense, though he lacked a proper definition of the message of the game. It seems that the realisation of the lack of agency needs more time to mature in the player. Some of the participants were confused about the fighting UI and some speculated that it might not have made a difference what they chose but still only one came to the full realisation. A couple of the participants even believed that some of the attacks were better than the other. To test our hypothesis, that the realisation of lack of agency results in a deeper perception of realism, we have to prolong the play session to include more fights or change the UI to make it easier to realise the lack of agency. That the design of the fighting UI is good at hiding the lack of agency correlates well with the notion of \citet{wardrip209agency} that the restriction of interaction can prevent the breakdown of the illusion of agency. We chose to use a UI that resembles that of the fights in the \textit{Pok�mon} gameboy games. A simple fight system UI restricted by being turn-based and having only a few options for actions. In the fights of our game we wanted to create the illusion that all the possible actions, if not possible now, will be possible later. For instance upon choosing to use an item the player will be told that they currently are not possessing any items implying that once you have collected an item you can use it. These illusion seems to work however we had thought that the participants would have realised the lack of agency sooner than they did.


What can we deduct from participant 9's experience of lack of agency if we suppose that his response would be the general response from players of the game. Upon realisation he knew that neither of his actions made sense other than pacing the speed of the fights. In another response he expressed that the game became boring towards the end even though he felt that the realisation made the purpose of the game shine through. For a future implementation of this game we have to consider how to weigh game purpose (partially conveyed through the realisation of lack of agency) against playability and continuation desire. 






IS DOG FIGHTING THE BETTER ALTERNATIVE?
One of the things we were curious about is whether dog fighting is perceived as a better alternative to prostitution. It is no secret that in our - the designers - mind dog fighting is a better alternative. Our test data indicate that the majority of the participants felt the same way but we cannot ignore the data that resists this notion. A trend in the data reveals that the visual appearance of the man you are to prostitute yourself to has had an impact on the choice. We might also have to question whether the choice is affected by the player's cultural background.

One participant(2) spend a minute making the decision. She was clearly emotionally affected by the game and the choice made her feel uncomfortable. Interestingly she was part of the first test session. In the implementation of the game used for the first session the man to which you can offer sexual favours is not depicted. In the second implementation of the game he is depicted as being very repulsive.  \commenting{Insert picture of man} The fact that he is not depicted in the first test session may have made the choice less unambiguous. This thought is supported by two other participants(7, 9) statements to the question "Why did you pick dog fighting over sexual favours?". They state that the man's uglyness played a role in their decision to pick dog fighting. A fourth participant(8) states that he felt the game wanted him to choose dog fighting over sexual favours. Whether the depiction of the man has had an impact on this can only be speculated but it seems like the new implementation of the game pushes the player more towards picking dog fighting over sexual favours.

Two(1, 6) of the three participants who chose sexual favours over dog fighting stated that they chose sexual favours out of curiosity. Observations of the participants, a male and a female, indicate that they were taking the seriousness of the game lightly and were having fun with it. They seemed to have a satirical distance to the game. The male participant(1) states that he chose the sexual favours because it was more extreme. The female(2) states that her choice may have been biased because she was sitting at a test. Their data indicate that they felt that dog fighting was the moral choice but that they chose sexual favours just to see what would happen in the game. Interestingly the last participant(4) chose sexual favours because she felt that it was the moral choice. Born and raised in Hong-Kong her cultural background is different from the rest of the participants (native Danes). Her reasoning being that no one dies in the sexual favours scenario. She is aware of, and believes that her opinion and choice is different from the other test participants. However she herself states that it is because of her cultural background.
