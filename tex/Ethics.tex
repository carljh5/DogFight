One of the main motivations behind this paper and game is the interest in figuring out the ethical boundaries of games, and whether those boundaries are different for games than for other art mediums. It is our hypothesis that the rules that apply for what is socially acceptable as subject matter in movies or books are not necessarily the same rules that apply for games. We believe a book about a dog fighter, would in no way cause the same amount of controversy and public outrage as a game where you play as a dog fighter. \

We could simply argue that it is because that games are viewed mainly as entertainment while movies and books are accepted art forms. That games right now are in the social state of films in the 1920's and will naturally evolve and become accepted as mediums of art. This does become a little too trivial an explanation and does not deal with the fact that games are by design a more engaging and embodying experience.\

In \citep{sicart2011ethics}, Sicart reasons that when playing a single-player computer game the player, phenomenologically, is the subject that experiences the game not as an object but as a process that interacts with and rewards the player. Here we find a fundamental difference between games as a medium and as abstract art since people would find it easier to describe a book as an objective experience, where the reader does not experience the book as interacting with them, but as an object that expresses the author as a subject. It is simply easier to deem something as not being unethical when you can see it completely as someones opinion or expression of art. With games it becomes much harder to accept them since they are viewed as a process. Hence the actions the game allow you to take is not only the expression of an authors morality but also the players which could potentially be you or someone you know. \

Many people also consider games to be a medium exclusively or mainly for kids. This would of course explain why those people have different ethical thresholds for games. \

The game series \textit{Grand Theft Auto} \citep{north2013grand} is one of, if not the, most popular examples of a game with moral controversies. \textit{Grand Theft Auto} is a game about car thievery and the killing of police officers and generic innocents. In the game you play the role of a gangster in a fictionalised city. The game is being referred to as a "murder simulator".\citep{murdersim} On the other hand fictional works like the movie \textit{Goodfellas} and the tv-series \textit{The Sopranos} deal with some of the same settings and themes. Both also include stylised violence yet they are still considered as works of popular culture art.\citep{sicart2011ethics} While there are games that include some level of violence, which are not met with the same caution, we still believe that this is an example of the difference in public perception of the mediums. The question then becomes how it would be possible to make a version of \textit{Grand Theft Auto} which would generally be considered ethically sound but still allow for some insight into the world of organised crime and its consequences.\

We theorise that realism and consequences could have some impact here. In \textit{Grand Theft Auto} breaking the law in front of a police officer means that he will try to arrest you. While this does simulate some level of real-life morality and consequences it also just serves as a gameplay mechanic. It is entirely possible to go on a killing spree and then escape the consequences by using a simple game mechanic like painting your car. It is possible that if the game had more realistic consequences it would not be considered unethical. If, for instance, you killed someone you would have to deal with a long trial and your grieving mother and your girlfriend leaving you while you rot in jail for 20 years. This could make the game more ethically acceptable while still allowing for the same affordance, but would of course increase the game severely in scope. Here other more linear mediums are less restricted since they can narratively explain why the character avoided (or dealt with) the consequences without needing to make it an abstraction. \

It is also important to note that rules have moral values so if we reward molesting a child it will give the game a certain (lack of) morality, same as if we reward the killing of gangsters or bad guys. If we arbitrarily reward the player for moral or immoral actions in the game it will create a moralising effect.\

In \textit{Super Columbine Massacre RPG} \citep{ledone2005super} the player is rewarded for completing actions they do not feel comfortable with. The player controls the two mass murderers during the Columbine massacre and the game does not leave you the choice of not committing the murders but instead rewards you mechanically for killing your fellow students and teachers. Most players will find playing this game very uncomfortable and will find a dissonance between being forced into committing these actions and being rewarded for them. While this seems inherently unethical and even more so than \textit{Grand Theft Auto} this discomfort does create an amount of artistic expression and as \citep{sicart2011ethics} states: \"This tension is crucial for understanding the potential of computer games as ethical experiences.\" \

It seems that a main problem with ethics exploration in games is when games allow for a wide range of possibilities without being able to model the consequences. Of course including all the consequences of murder in a game like \textit{Grand Theft Auto} would be non-sensical since the player would still be aware that she is playing a game. She would simply be able to turn off the game if she does not want to suffer the consequences.\

If we were able to create a complete single-player life simulator that is a 1-to-1 adaption of real-life with only the one change being that you are aware that this is a game, and would be able to turn it off, it would probably not be well-received. It could bring out extremely destructive or libertine behaviour in people when playing the game which would accordingly bring public outrage of allowing those actions in a simulation. It does of course become interesting both philosophically, politically and artistically whether we should allow for these explorations of the bad parts of human behaviour.\

We are interested in researching, if linearity allows for a greater artistic expression and possibilities of dealing with unethical affordances in an ethical manner? \

\subsection{The Background Story}
\label{story}
In \textit{Grand Theft Auto} the story for each game normally starts with the main character trying to escape his troubled criminal past and begin a new law-abiding life but is dragged back into the criminal life as his only method of survival and prosperity. This way of presenting the illegal life as the only and necessary choice is a classic narrative tool for justifying or at least sympathising with the actions of a character. \textit{"Ethos anthropoi daimon"} wrote Heraclitus \- \textit{"character is fate"}, meaning that you cannot escape your background.\ 

Another narrative tool for generating sympathy is having the unethical choice be the lesser of two evils. In the movie \textit{Natural Born Killers} \citep{stone1994natural} we follow two serial killers, Mickey and Mallory, in a very graphic and stylised cross-country slaughter. The film presents Mallory as being sexually harassed by rednecks and abused by her family, which is also the first victims in the film. The cop that pursues them is also a sadistic psychopath. These malevolent characters help, if not justify the actions, at least create some initial sympathy and understanding towards the main characters without which the movie would have been cynical and hard to watch. \

For the dog fighting game we will create a back story which tries to use both of these narrative tools. Over 15.000 homeless kids under the age of 18 live in Mexico City. They live in the parks and sewers of the city many are addicted to solvent and have to prostitute themselves to earn money for food \citep{baverstock}. The main character of our game is one of these so called \textit{rat kids}. By using this background we establish a situation where unethical ways of living are his fate and also make room for a great deal of compassion for his situation and actions. \

Furthermore, we will set up the choice of starting participating in dog fighting in a way so it becomes an active choice for the player but still a choice between two evils where dog fighting hopefully is perceived as the lesser evil. We frame it such that the main character will meet another street kid who is in the process of catching a dog to use in fighting. When the other street kid suggests that the player character starts doing the same an older man meets them and solicits the main character for prostitution. The player can then choose how the main character should earn his money by \textit{dog fighting} or \textit{sexual favours}. This condenses the seminal life choice into a single pregnant moment.\

This should hopefully justify the action of starting dog fighting and also make it a conscious player choice.\
