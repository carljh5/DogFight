In \citep{wardrip2009agency} \textit{Player agency} is defined as the overlapping set of what the player desires to do in a game and what is supported by the underlying computational model.\

We define \textit{Player agency} as: When a player is able to (1) control their own or their characters actions, (2) have these actions matter in the game world and (3) have sufficient information to expect the outcome of these actions. The less player agency there is in a game, the more procedural the game is. E.g. a traditional book in one end would have no agency and be completely procedural, while a table-top role-playing game controlled by a Game Master, could allow for a very high amount of player agency, depending on the flexibility of the Game Master.\

\citep{wardrip2009agency} also covers how the presence of an understandable user-interface (UI) can make players feel like they have more agency and thereby create an illusion of agency. The breakdown of this illusion of agency will then lead to the player becoming frustrated and stop playing.\

WF - the restriction of interaction can prevent the breakdown of the illusion of agency\
\commenting{mention this later, in regards to why the players fall to recognize the loss of agency}\

\commenting{mention the superstition of small actions having an influence on a seemingly unrelated outcome. e.g. blowing on dice before rolling them or the more direct of clicking the ball when catching pokémon in pokémon go}\


We expect the players to lose interest in interacting with the battle system, once they figure out that the agency is an illusion. Still, we hope that they will be able to transcend the message of impotence in such an important situation, where you could potentially lose a dog. 
We hope this will actually make the players become involved and have a realistic experience of dog fighting. \

It is important to not that we do not consider this whole concept of UI allowing for impotent options to be a game mechanic, but an artistic instrument meant to create reflection and empathy in the player. \
