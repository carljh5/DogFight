
\documentclass[preprint,12pt, authoryear]{elsarticle}

%% Use the option review to obtain double line spacing
%% \documentclass[preprint,review,12pt]{elsarticle}

%% Use the options 1p,twocolumn; 3p; 3p,twocolumn; 5p; or 5p,twocolumn
%% for a journal layout:
%% \documentclass[final,1p,times]{elsarticle}
%% \documentclass[final,1p,times,twocolumn]{elsarticle}
%% \documentclass[final,3p,times]{elsarticle}
%% \documentclass[final,3p,times,twocolumn]{elsarticle}
%% \documentclass[final,5p,times]{elsarticle}
%% \documentclass[final,5p,times,twocolumn]{elsarticle}


\setlength\parskip{\baselineskip}

%% The graphicx package provides the includegraphics command.
\usepackage{graphicx}
%% The amssymb package provides various useful mathematical symbols
\usepackage{amssymb}
%% The amsthm package provides extended theorem environments
%% \usepackage{amsthm}

%to color text
\usepackage{xcolor}

%% The lineno packages adds line numbers. Start line numbering with
%% \begin{linenumbers}, end it with \end{linenumbers}. Or switch it on
%% for the whole article with \linenumbers after \end{frontmatter}.
\usepackage{lineno}

%%to allow for ø and é
\usepackage{inputenc}


\usepackage{hyperref}

\usepackage{csquotes}

%% natbib.sty is loaded by default. However, natbib options can be
%% provided with \biboptions{...} command. Following options are
%% valid:

%%   round  -  round parentheses are used (default)
%%   square -  square brackets are used   [option]
%%   curly  -  curly braces are used      {option}
%%   angle  -  angle brackets are used    <option>
%%   semicolon  -  multiple citations separated by semi-colon
%%   colon  - same as semicolon, an earlier confusion
%%   comma  -  separated by comma
%%   numbers-  selects numerical citations
%%   super  -  numerical citations as superscripts
%%   sort   -  sorts multiple citations according to order in ref. list
%%   sort&compress   -  like sort, but also compresses numerical citations
%%   compress - compresses without sorting
%%
%% \biboptions{comma,round}

% \biboptions{}

% remove preprint from foot
\makeatletter
\def\ps@pprintTitle{%
 \let\@oddhead\@empty
 \let\@evenhead\@empty
 \def\@oddfoot{}%
 \let\@evenfoot\@oddfoot}
\makeatother


\newcommand\commenting[1]{\textcolor{red}{\textbf{\underline{#1}}}}

\begin{document}

\begin{frontmatter}

%% Title, authors and addresses

\title{Do Dog Fighters Love Their Dogs?}

%% use the tnoteref command within \title for footnotes;
%% use the tnotetext command for the associated footnote;
%% use the fnref command within \author or \address for footnotes;
%% use the fntext command for the associated footnote;
%% use the corref command within \author for corresponding author footnotes;
%% use the cortext command for the associated footnote;
%% use the ead command for the email address,
%% and the form \ead[url] for the home page:
%%
%% \title{Title\tnoteref{label1}}
%% \tnotetext[label1]{}
%% \author{Name\corref{cor1}\fnref{label2}}
%% \ead{email address}
%% \ead[url]{home page}
%% \fntext[label2]{}
%% \cortext[cor1]{}
%% \address{Address\fnref{label3}}
%% \fntext[label3]{}


%% use optional labels to link authors explicitly to addresses:
%% \author[label1,label2]{<author name>}
%% \address[label1]{<address>}
%% \address[label2]{<address>}

\author{Carl Johan Hanberg}
\author{Paw H\o vsgaard Laursen}


\begin{abstract}
%% Text of abstract
Lorem ipsum.Lorem ipsum.Lorem ipsum.Lorem ipsum.Lorem ipsum.Lorem ipsum. Lorem ipsum.

Lorem ipsum.Lorem ipsum.Lorem ipsum.Lorem ipsum.Lorem ipsum.Lorem ipsum.Lorem ipsum.Lorem ipsum.Lorem ipsum.

Lorem ipsum.Lorem ipsum.Lorem ipsum.Lorem ipsum.Lorem ipsum.Lorem ipsum.Lorem ipsum.
Lorem ipsum.Lorem ipsum.Lorem ipsum.Lorem ipsum.Lorem ipsum.Lorem ipsum.Lorem ipsum.

\end{abstract}

%%\begin{keyword}
%%Science \sep Publication \sep Complicated
%% keywords here, in the form: keyword \sep keyword

%% MSC codes here, in the form: \MSC code \sep code
%% or \MSC[2008] code \sep code (2000 is the default)

%%\end{keyword}




\end{frontmatter}

\tableofcontents
\clearpage

%% main text
\section{Introduction}
\blockquote{
"If nothing else makes Mr. London's book popular, it ought to be rendered so by the complete way in which it will satisfy the love of dog fights apparently inherent in every man."

- Quote from review of \textit{"Call of the Wild"} \citep{london1997call} from \textit{New York Times}, 1903 \citep{barnesnoble}
}

The above quote tells of a time, where dog fighting was not only romantizised, but also glorified. Many earlier works of litterature share this depiction of staged fights between or against animals, as something inherently aestethic and masculine. \

Recently \textit{Guggenheim Museum} in New York decided to pull three major works from a highly anticipated exhibition after pressure from animal-rights supporters.\citep{guggenheim} One of the works was a seven-minute video of eight American Pit Bull Terriers, which are mostly known for their aggressiveness and popularity in illegal dog fighting. In the video the dogs are opposite each other, fixated on thread mills, exhausting themselves trying to reach, and presumably maul, the others. An online petition against the art works was signed by 600.000 and after receiving several threats of death and violence Guggenheim pulled the works �out of concern for the safety of its staff, visitors and participating artists.�.\

The boundaries for what is socially acceptable are always changing, but it becomes both a democratic and cultural problem, when established cultural institutions have to pull works from fear of reprecussions. Art in its essence needs to be able to deal with the hard topics and dilemmas of our society.\ 

For video games the boundaries even tighter. Video games are often not even considered an art form, and have even stricter rules for what subjects they are allowed to examine. \

We live in a world, where cruelty against animal and people is happening every day. The arts need to be able to examine the reasons and culture behind even these actions. Video games are in their definition able to put the player in the place of unknown characters and thereby convey feelings and settings that would otherwise be completely alien to the player.\

We are interested in determining, whether we can create a game about street kids and dog fighting without the game becoming neither unethical nor moralizing.\

\subsection{Problem Statement}
\label{ProbStat}

Dog fighting is a brutal blood sport, which often results in serious injuries or death of the participating dogs. While it is illegal in most countries, with Japan, Honduras and Russia being the exceptions, it is still practiced illegally in many countries for entertainment and betting purposes. \

Violence and other illegal subject matter in video games, while being the cause of many controversies, are generally commonplace in many popular video games. While the causality between video game violence and aggressive behaviour can be questioned \cite{boxer, ferguson} there exist no literature to our knowledge that proofs a causality between playing games with illegal subject matter and committing those types of crimes. \

Despite having inherent game mechanics, dog fighting is not generally used as subject matter. We expect this to be because of the social norms and developers�� fears of public outrage or ostracisation. \

In this project, we will analyse how to use dog fighting as subject matter in games. Inspired by \cite{sicart2011ethics}, we are interested in examining whether unethical affordances is able to exist in a game with a purpose, as well as how player agency, or the lack thereof, will contribute to the players frustration over the loss of their dogs. \

We will work with the following Research Questions:

\begin{itemize}
\item RQ1. Can we create unethical affordances, namely dog fighting, in a game without making it an unethical game?
\item RQ2. Will the player be able to feel empathy for the dogs, while being reliant on them fighting?
\item RQ3. How does the player feel, when they lose their dogs, while lacking the agency to prevent it?
\end{itemize}

\subsection{Methods}
The project will include an overview of the field of dog fighting games, as well as a case study of the ethics of computer games. \

We will produce the mobile dog fighting game ��Rat Kid Dog Fight!��. We will conduct extensive playtests, focusing on the reactions of players in particular game situations.\

The main character in the game will be portrayed as one of the 15.000 homeless children of Mexico City \citep{baverstock}, which tries to break the circle of drug abuse and prostitution, by training stray dogs to battle in the illegal dogfights of the local gangs. This narrative will make dog fighting the only viable option for the character and will be presented as a lesser of two evils, which can help the players justify their own actions. \

The gameplay mechanics will draw similarities to the Pokemon game series \citep{game:pokemon} and other pet management games, where the player is also afforded the option of pitting animals against each other. While we are interested in having minimal player agency in the dog fights, the Pokemon similarities should create the expectancy of player agency \citep{wardrip2009agency}. \

We will experimentally determine how this loss of player agency in the battles will affect the player��s reactions to the death of their dog. The end result being an investigation of how gameplay elements and game affordances are able to imminently immerse the player, but also how it can make them reflect on their personal morality about the subject matter.\

Finally, we will determine from the prototype and test data how we can best continue development towards a full-featured mobile game, and if such a game would be best suited for retail distribution or art exhibitions.\

\section{Ethics in Games}
\label{Ethics}
Explain the boundaries of ethics in games. \

Postal and Columbine reactions and references\

The lines are not the same as for other art forms. Reference. In other arts the actual hurting of animals are necessary for outrage\

Why? Because games are more entertainment, not art? because of the embodiedment of players/ the subjective point of view? We chose to do something so it becomes up to us to define whether this fits our moral, while we can understand a film or book as someone elses narrative and moral. If true, this also positions games as a more potent medium for provocative art. \ 

\subsection{The Background Story}

Rat Kid Background / justification for unethical affordances\

Character makes fate (Heraclitus, gta, Miguel), we used the same to allow these affordances \

interested in testing when the game is perceived as unethical\


\section{The Breakdown of the Illusion of Agency}
\label{Agency}
In \citep{wardrip2009agency} \textit{Player agency} is defined as the overlapping set of what the player desires to do in a game and what is supported by the underlying computational model.\

We define \textit{Player agency} as: When a player is able to (1) control their own or their characters actions, (2) have these actions matter in the game world and (3) have sufficient information to expect the outcome of these actions. The less player agency there is in a game, the more procedural the game is. E.g. a traditional book in one end would have no agency and be completely procedural, while a table-top role-playing game controlled by a Game Master, could allow for a very high amount of player agency, depending on the flexibility of the Game Master.\

\citep{wardrip2009agency} also covers how the presence of an understandable user-interface (UI) can make players feel like they have more agency and thereby create an illusion of agency. The breakdown of this illusion of agency will then lead to the player becoming frustrated and stop playing.\

WF - the restriction of interaction can prevent the breakdown of the illusion of agency\
\commenting{mention this later, in regards to why the players fall to recognize the loss of agency}\

\commenting{mention the superstition of small actions having an influence on a seemingly unrelated outcome. e.g. blowing on dice before rolling them or the more direct of clicking the ball when catching pokémon in pokémon go}\


We expect the players to lose interest in interacting with the battle system, once they figure out that the agency is an illusion. Still, we hope that they will be able to transcend the message of impotence in such an important situation, where you could potentially lose a dog. 
We hope this will actually make the players become involved and have a realistic experience of dog fighting. \

It is important to not that we do not consider this whole concept of UI allowing for impotent options to be a game mechanic, but an artistic instrument meant to create reflection and empathy in the player. \


\section{Game Design Considerations}
\label{Design}
In this section we will go through which design considerations we had when making the game.\

\subsection{Design Goals}
\label{designGoals}

\begin{figure} 
	\centering
    \includegraphics[width=0.5\textwidth]{pitbull.jpg}
    \caption{American Pit Bull Terrier}
    \label{fig:pitbull}
\end{figure}

One of the design goals of the game is to have a battle system that can both be extremely deadly, where any bite could potentially kill, and also include painfully long-drawn-out matches determined by endurance. It is meant to be a stone-paper-scissors-like system, where a very aggressive dog (like an American Pit Bull Terrier seen in Figure \ref{fig:pitbull}) would be able to beat a stronger dog (like the Japanese fighting dog Tosa Inu seen in Figure \ref{fig:tosa}), but lose to a more courageous dog (like the Russian shepherd dog Caucasian Ovtcharka seen in Figure \ref{fig:ovtcharka}\footnote{Often used in dog fighting, the Ovtcharka is genetically bred to guard a flock of sheep - against a pack of wolves - on its own hence it is incredibly brave}), but the courageous dog would then lose to the stronger dog. We do not have any knowledge whether this is how the fights between these dogs would turn out in reality, even if this is something that is wildly discussed on internet fora.\

\blockquote{"You want the truth don’t you? Well here goes. I’m 65 years old and fought dogs against the likes of Don Mayfield, Don Maloney, Walter Komosinski, Pete Sweeney, Earl Tudor, Jack Kelly, Jerry Bean, I could go on but what’s the point? I won my share and lost my share. IF there was a better combat dog than the pit bull, then that’s what the pros would be using. You’re talking about a 175 lb. dog against a 50 lb pit bull. The pit bull stil wins. Why? Because the Russian dog WASN’T bred for combat. I’ve seen pit bulls up against everything available and they still win. Are there exceptions to every rule? Of course. When you match a 175 dog against a 50 lb pitbull, the odds are stacked against the smaller dog. But, pound for pound there is NO dog capable of whipping a pit bull. As a matter of fact, I will put a 50 lb. pit bull up against any 175 lb Russian dog in the world. \

There’s an old saying that goes like this, “it’s not the size of the dog in the fight, but the size of the fight in the dog.” The American Pit Bull is the world’s greatest combat animal ever bred. If you think placing a 175 lb. Russian dog against a 50 lb. pit bull you have my deepest sympathies. When the pit bull grabs the big dog by the now and won’t let go, let’s see how much screaming goes on fro the big dog. And make no mistake, the pit bull IS gonna at some time or another bite the bigger dog in a place he ain’t gonna care much for. And once again, if the “pros” who fight dogs felt like the Russian dog was the best combat dog, don’t you think they’d be using them instead of the lowly pit bull? ‘Nuff said." \citep{amptvsovc}\label{pitbullQuote}}\


The above quote examplifies how wide-spread this discussion of "Which dog would win in a fight?" is in specific circles. That this is discussed in such a manner also tells us that dog fighting as subject matter in games, might be very appealing to some people, even if finding out the popularity of such a game is not the point of this study. 

\begin{figure} 
	\centering
    \includegraphics[width=0.5\textwidth]{tosa.jpg}
    \caption{Tosa Inu}
    \label{fig:tosa}
\end{figure}

In the prototype the first fight is hard-coded to be a win for the player and the fourth match will always be a loss. So if the \textit{wrong} dog were to kill the other and win, it would instead miss. This is done to streamline the testing session.\

\begin{figure}
	\centering
    \includegraphics[width=0.5\textwidth]{ovtcharka.jpg}
    \caption{Caucasian Ovtcharka}
    \label{fig:ovtcharka}
\end{figure}

\subsection{Game Features}
The game prototype we created has the following features:

\begin{enumerate}
	\item iOS Mobile Application
	\item Text story
	\item Events with player choice
	\item Event and character pictures with simple animations and sounds
	\item Map, where you can select where to go (Only on place to choose from in the prototype)
	\item Battle system where dogs battle eachother
	\item Ability to train dogs
	\item Ability to catch dogs (Will never succeed in the prototype)
	\item Option to breed dogs (Will never succeed in the prototype)
\end{enumerate}

\begin{figure}[h!] 
	\centering
    \includegraphics[width=0.5\textwidth]{GameScreen1.png}
    \caption{Story screen from the \textit{Rat Kid Dog Fight!} prototype}
    \label{fig:GameScreen}
\end{figure}

%why did we choose which features?

We chose a vertical (instead of horizontal) display for the phone for accessibility and to signal the briefness of the play session.

\subsection{Covering Prototype Limitations}
\label{limitations}
While we were creating a limited and deterministic prototype, we made several considerations, to hide the procedural and deterministic nature of the prototype. These include the map system, the ability to catch and breed, which all speak to a bigger game than the prototype is. The reason for doing this was to create an illusion of a longer game, and a potentially longer relationship with the dog. We feared that if the player knew the scale of the prototype or the already determined outcomes, they would not allow themselves to become emotionally attached to the dog. \

To create this illusion the UI is not build for simply supporting the features of the prototype, but also to support the affordances a longer and less procedural game could include. The option to 'Breed' the dogs is a feature that is not actually included in the prototype, but since the player is forced to only catch dogs of one specific gender, they will never discover this.\

\subsection{Lack of Player Agency in Fights}
As mentioned in chapter \ref{Agency} we have implemented impotent options in the dog fights, which aims to create the illusion of agency. These options mimic the \textit{Pok\'emon} games, as seen in figure \ref{fig:PokeBattle} and \ref{fig:DogFightBattle}. \

\begin{figure}[h!] 
	\centering
    \includegraphics[width=0.5\textwidth]{PokemonBattle.png}
    \caption{A \textit{Pok\'emon} battle}
    \label{fig:PokeBattle}
\end{figure}

\begin{figure}[h!]
	\centering
    \includegraphics[width=0.5\textwidth]{battle.png}
    \caption{A \textit{Rat Kid Dog Fight!} battle}
    \label{fig:DogFightBattle}
\end{figure}

All these battle options does not have any effect on the actual battle. The \textit{Fight} option let the player select between \textit{Throat Bite}, \textit{Tackle}, \textit{Scratch} and \textit{Lock Bite}. While \textit{Scratch} and \textit{Tackle} are typical generic \textit{Pok\'emon} moves, the other two also resemble names of moves from \textit{Pok\'emon} with a dog fight connotation, which also makes them appear more visceral and brutal. When chosen, all of these fight-options simply give the text feedback of your character shouting something appropiate, like "Go for the throat, DOGNAME!" for selecting \textit{Throat Bite}, which also is a mediation of the television series \textit{Pok\'emon: The Series} (1997), where the main character, \textit{Ash}, could shout "Pikachu, use Thunderbolt!". This reference further creates the expectance of an actual effect on gameplay, since those familiar with the show, would know that there is always a direct link between what the Pok\'emon trainer shouts and what the Pok\'emon does and would expect the same causality here.
The option \textit{item} simply states that the player has no items to use and then progresses the battle. Likewise, the option \textit{Dog} states that the player "... cannot change dogs during this fight.". Similar to the (lacking) features mentioned in chapter \ref{limitations}, this gives the player an idea of a bigger game, where they might unlock these options later on.\

The goal with creating this illusion of player agency is to create a feeling of helplessness and hopefully the realization of no agency will put them in the place of the kid. \ 
We want to battles to be realistically chaotic and brutal, while still giving the player the illusion of being able to change the inevitable, creating the same superstition as mentioned in chapter \ref{Agency}.\

\subsection{Battle System}
\label{battleSystem}
Even though the player has limited agency in the fight system, we still simulate the fights with a very game-like system.\

\begin{figure}[h!]
	\centering
    \includegraphics[width=0.5\textwidth]{DogStats.png}
    \caption{A dog stat screen from the \textit{Rat Kid Dog Fight!} prototype}
    \label{fig:DogStatScreen}
\end{figure}


The dogs has the following 5 stats: \textit{Aggression}, \textit{Strength}, \textit{Courage}, \textit{Speed} and \textit{Bite}.\

To get an overview of the battle system please see Table \ref{tab:stat}.

\begin{table}[h]
\scriptsize 
	\begin{tabular}{|l|p{5cm}|p{3cm}|p{3cm}|} 
		\hline
		\textbf{Game Round} & \textbf{Description} & \textbf{Example Outcome} &\textbf{ Feedback }\\ [0.5ex] 
		\hline\hline
		Aggression Roll & Each dog has a an aggression score. The one with the highest aggression will decrease the other dogs strength speed and bite depending on the others courage rating & Dog1s 10 AGGRESSION goes against Dog2s 6 courage and decreases Dog2s STRENGTH, SPEED and BITE by 40\% & "Dog2 is intimidated by the vicious barking from Dog1" \\  
		\hline\hline
		\textit{\textbf{Main loop}} \\
		\hline
		Speed Roll	& The dog with the highest roll, determined by speed bites first. &	Dog1 goes first & "Dog1 snaps after Dog2s neck. \\
		\hline
		Bite roll & The biting dogs rolls BITE to penetrate to kill or injur the other dog. There is always a 20\% chance to miss. Otherwise the bit will reduce the other dogs STRENGTH by the attacking dogs BITE divided by three. If it hit, the attacking dog has a 30\% chance to get its jaw to lock on the other dog. & Dog1 bites Dog2 & "Dog1 Bites Dog2. Its jaws locking onto the skin of Dog2s neck." \\
		\hline
		Other dogs bite Roll & Similar to above. & & "Dog2 snaps after Dog1. Dog2 misses Dog1." \\
		\hline
	\end{tabular}
	\caption{\label{tab:stat}A tabular representation of the battle system}
\end{table}


The main loop continues until one of the dogs has 0 \textit{Strength}, after which that dog dies.\

Both \textit{Speed} and \textit{Bite} are relative to the current \textit{Strength} of the dog, so their effectiveness decreases, as the fight goes on. \textit{Bite} will always be at a minimum of 50\% of the \textit{Bite} stat. This also means that a strong dog will be less decreased on other attributes, as the fight goes on. While a dog has its bite locked around the other one it does not bite the other dog, instead it just decreases the opposing dog's \textit{Strength} by a small margin. The other dogs chance to bite the dog with the locked bite is also decreased.\

If the jaws of both dogs are locked\footnote{Upon biting each other the dogs may lock their jaws to keep a hold of each other. See  \url{http://thetruthaboutpitbulls.blogspot.dk/2012/06/locking-jaws.html} [Retrieved: December 13, 2017].} onto the skin of each other the organisers will separate the dogs with steel bars. This is a big part of the natural narrative of real dog fighting, as mentioned in \citeauthor{london1997call}'s \textit{Call of the Wild} \citeyearpar{london1997call}, where the dogs' jaws are separated with gun barrels for dramatic effect.\

\subsection{Actions Between Fights}
Between the dog fights the players had the option of either training their dogs, breeding the dogs or catching a new dog. As mentioned in chapter \ref{limitations} both catching other dogs or breeding is not possible to succeed in the prototype and are only there to give the illusion of a more expansive game. The training is done by rapidly tapping the screen to get the bar seen in Figure \ref{fig:training} full up. Then the dog's respective stat increases by one. If the bar becomes completely empty before this, the training fails and nothing happens.\ 

\begin{figure}[h!] 
	\centering
    \includegraphics[width=0.5\textwidth]{Training.png}
    \caption{Training screen from the \textit{Rat Kid Dog Fight!} prototype}
    \label{fig:training}
\end{figure}

By only allowing one choice per game week, we implicitly associate a cost with the action, and an investment in your dog. This also frames the game loop nicely, so the main loop becomes Train/Catch/Breed \rightarrow Dog Fight \rightarrow Train/Catch/Breed \rightarrow ...

\section{Testing}
\subsection{Method}
To answer our hypotheses we have designed a qualitative play experience test. The procedure of the test includes a playthrough of the game followed by a semi-structured interview. The in-game choices made by the player in the playthrough is recorded as is their body language, statements or other sounds produced while playing.

The test subjects will be asked to participate in one playthrough of the game followed by a semi-structured interview to gather data on that specific play experience. Due to the mature content of the game the targeted test subjects are age 18 or above. To look for trends across genders we aim for gender equality in our sample. Furthermore while recruiting test subjects we look for subjects who has outspoken opinions about dogs, animal rights and/or human rights.

As mentioned in chapter \ref{designGoals} the prototype is hard-coded so that the player will always win at least one fight and loss the fourth fight. This makes it possible for us to determine how the testers react when their dog kills another and also when their own dog dies.\

\subsection{Sessions}
Two testing sessions were conducted on the IT University of Copenhagen. The first on the 30th October 2017, the second on the 23rd November 2017. The test procedure followed the same guidelines for both of the tests however a few of the questions were rephrased and the implementation of the game underwent several changes. The changes included correction of spelling in the narrative, more graphical elements depicting the situation of the state of the game and better feedback in the dog fights in the game. These changes will have biased the test results and should be kept in mind when analysing the data.

A playthrough of the game took 10-15 min. depending on the speed of the participant - reading, taking decisions - and the random outcomes of the fights in the game. While playing the participants in-game choices was recorded. Some participants came to an end of the game after the second fight of the game others after the third.
The interview was conducted following af semi-structured questionnaire consisting of 17 predefined questions, participant demographics questions and a small observational sheet to record the actions taken by the participant in the game. Furthermore observations of the participants body language, statements and other kinds of sounds while playing the game was recorded.

\subsection{Results} \label{Results}
In total nine subjects, four male and five female, successfully participated in the test. The youngest being 24 and the oldest 31, average being 27,2 years old. Six of the participants are higher education students, two are film directors and the last participant is educated but unemployed. It is safe to conclude that the participants on average possesses intellectual judgement. In this chapter (\ref{Results}) the data obtained from the interviews and the play test with the participants are presented. The data has been coded and analysed for relevant trends. First observations of the players decisions in-game are presented. Next, each question are presented along with a highlight of the participants responses to either the question or a general trend found in the data to that question. The participants are represented by number (1-9). Participants 2, 4, 6 and 7 are female, 1, 3, 5, 8 and 9 are male. 

\subsubsection{In-game Decisions} \label{ingamedec}

\textbf{Sharing glue.} The first in-game decision the player has to make is whether to share the sniffing glue or not. The majority of the participants shared the glue. Participant (4, 7, 9) chose not to. (Table \ref{tab:glue}).

\begin{table}[h]
\centering
\begin{tabular}{l l l}
\hline
\textbf{Shared the glue?} & Yes & No \\
\hline
Male & 4 & 1 \\
Female & 2 & 2 \\
\textbf{Total} & 6 & 3 \\
\hline
\end{tabular}
\caption{\label{tab:glue}Overview of participants in-game decisions on whether to share the sniffing glue or not}
\end{table}


\textbf{Chosen protagonist name.} The second decision the participants had to make in the game was to give the protagonist a name. Six participants (1, 2, 3, 4, 5, 7) chose to give the protagonist a, what we define as a, serious name (Table \ref{tab:name}). The serious names are names that resemble the participants real name or is a known nickname of theirs while the non-serious names are not. We speculate that the participants chose a serious name because they want to identify with the protagonist rather than distance themselves from him. While picking a non-serious name is a way to create a humorous distance between yourself and the protagonist in the game.

\begin{table}[h]
\centering
\begin{tabular}{l l l}
\hline
\textbf{Protagonist name} & Serious & Non-serious \\
\hline
Male & 3 & 2 \\
Female & 2 & 2 \\
\textbf{Total} & 5 & 4 \\
\hline
\end{tabular}
\caption{\label{tab:name}Overview of participants choice of seriousness of protagonist name}
\end{table}


\textbf{Destiny and duration of decision.} The most important decision of the game is when the player is asked to choose their destiny. Whether they will participate in dog fighting or offer sexual favours to earn money. The majority of the participants chose dog fighting, only three (1, 4, 6) chose sexual favours (Table \ref{tab:dest}). All but one of the participants made the choice relatively quickly between three to nine seconds. The one exemption, participant 2, spend a minute before choosing dog fighting.

\begin{table}[h]
\centering
\begin{tabular}{l l l}
\hline
\textbf{Chosen destiny} & Sexual favours & Dog fighting \\
\hline
Male & 1 & 4 \\
Female & 2 & 2 \\
\textbf{Total} & 3 & 6 \\
\hline
\end{tabular}
\caption{\label{tab:dest}Overview of participants in-game decisions on the destiny of the protagonist}
\end{table}


\textbf{Choice of dog.} Boxer was the favourite choice of dog among the participants. Only one participant (7) chose the Golden Retriever. Interestingly the Husky was the favourite among the male participants (1, 3, 8) while no female participants chose it (Table \ref{tab:dog}).

\begin{table}[h]
\centering
\begin{tabular}{l l l l}
\hline
\textbf{Choice of dog} & Boxer & Husky & Golden Retriever \\
\hline
Male & 2 & 3 & 0 \\
Female & 3 & 0 & 1 \\
\textbf{Total} & 5 & 3 & 1 \\
\hline
\end{tabular}
\caption{\label{tab:dog}Overview of participants choice of dog}
\end{table}


\textbf{Chosen dog name.} The chosen dog names has, like the chosen protagonist names, been divided into serious and non-serious names. Now only two participants (2, 7) chose a serious name (Table \ref{tab:dame}). 

\begin{table}[h]
\centering
\begin{tabular}{l l l}
\hline
\textbf{Dog name} & Serious & Non-serious \\
\hline
Male & 0 & 5 \\
Female & 2 & 2 \\
\textbf{Total} & 2 & 7 \\
\hline
\end{tabular}
\caption{\label{tab:dame}Overview of participants choice of seriousness of dog name}
\end{table}


\textbf{Leashing dog.} Only two participants (6, 8) chose to leash their dog (Table \ref{tab:leas})

\begin{table}[h]
\centering
\begin{tabular}{l l l}
\hline
\textbf{Leashed the dog?} & Yes & No \\
\hline
Male & 1 & 4 \\
Female & 1 & 3 \\
\textbf{Total} & 2 & 7 \\
\hline
\end{tabular}
\caption{\label{tab:leas}Overview of participants in-game decisions on whether to leash their dog or not}
\end{table}


\textbf{Weekly options.} Unfortunately the decision of the weekly option (see chapter \ref{weekly}) was not recorded for the participants (1, 2) in the first test session. However in the second test session all of the participants chose to train their dogs.


\subsubsection{Interview} \label{interview}
\textbf{Q1.} A trend in the respones to the question "How would you characterise the experience?" is the participants description of the emotional impact the play experience had on them, see Table \ref{tab:emo}. The majority of the participants felt a negative emotional impact. Only one participant (6) does not express any emotional relation to the play experience. Her response implies that there was too much text in the game for her to be engaged.

\begin{table}[h]
\centering
\begin{tabular}{l l l l}
\hline
\textbf{Emotional impact from}\\
\textbf{play experience?} & Negative & Positive & Other \\
\hline
Male & 3 & 2 & 0 \\
Female & 3 & 0 & 1 \\
\textbf{Total} & 6 & 2 & 1 \\
\hline
\end{tabular}
\caption{\label{tab:emo}Emotional impact described in the responses to the question "How would you characterise the experience?"}
\end{table}

\textbf{Q2.} The responses to the question "How did you like your character?" are diverse and raises an interesting question; what is the participants perceived relation to the protagonist in the game? A trend in the data shows that the majority of the participants feels sympathy for the protagonist and/or his situation in the game (Table \ref{tab:symp}). Interestingly only two of the female participants (2, 7) express sympathy for the protagonist which raises the question whether the gender of the protagonist (male) has an influence on the felt sympathy towards him. However studying the responses it becomes clear that the participants has different interpretations of their relationship as players towards the protagonist. Participant (4) does not really answer the question but instead respond that it is just her self making decisions, hence she did not have any emotional connection to the protagonist. Participant (3) on the other hand states that the sparse description of the protagonist enabled him to project his self onto the protagonist. A third participant (8) states that he think the kid is sweet and that he want to follow him around, implying that he does not control the protagonist but merely the circumstances in which the protagonist is placed.

\begin{table}[h]
\centering
\begin{tabular}{l l l l}
\hline
\textbf{Sympathy for the}\\
\textbf{protagonist?} & Yes & No & Other \\
\hline
Male & 5 & 0 & 0 \\
Female & 2 & 2 & 0 \\
\textbf{Total} & 7 & 2 & 0 \\
\hline
\end{tabular}
\caption{\label{tab:symp}Expression of sympathy towards the protagonist in the responses to the question "How did you like your character?"}
\end{table}

\commenting{Man kan godt tage en moralsk beslutning selvom begge valg er uetiske}
\textbf{Q3.}  As can be seen in Table \ref{tab:choice}, the majority of the participants expressed that the moral option was to choose dog fighting over sexual favours in the game. Only one participant (4) felt sexual favours was the right option, but elaborates that her ethics might be different than the rest of the testers because of her cultural background. Participant 4 is born and raised in Hong-Kong as opposed to the rest of the testers who's ethnicity is Danish. Participant 8 chose dog fighting because he felt that the game wanted him to do so and participant 9 chose dog fighting because the man to which you can offer sexual favours was too repulsive. Participant 7 also mentions the man's appearance as one of the deciding factors for her choice.

\begin{table}[h]
\centering
\begin{tabular}{l l l l}
\hline
\textbf{Sexual favours or dog fighting,}\\
\textbf{which is the moral option?} & Sexual favours & Dog fighting & Other \\
\hline
Male & 0 & 3 & 2 \\
Female & 1 & 3 & 0 \\
\textbf{Total} & 1 & 6 & 2 \\
\hline
\end{tabular}
\caption{\label{tab:choice}Expression of moral in responses to the question "Why did you pick sexual favours or dog fighting over the other?"}
\end{table}

\textbf{Q4.} Judging from the responses to the question "Did you feel uncomfortable making the choice?" most of the participants were comfortable making their decision(Table \ref{tab:unco}). Unfortunately participant 2's response to the question is lost, but the observations recorded at her play session indicate that it was a very difficult choice for her. Apart from her body language signifying her being uncomfortable, she spend approximately one minute deciding which option to choose. In contrary the rest of the participants spend between 3-9 seconds. 

\begin{table}[h]
\centering
\begin{tabular}{l l l l}
\hline
\textbf{Did you feel uncomfortable}\\
\textbf{making the choice?} & Yes & No & Other \\
\hline
Male & 1 & 4 & 0 \\
Female & 2 & 2 & 0 \\
\textbf{Total} & 3 & 6 & 0 \\
\hline
\end{tabular}
\caption{\label{tab:unco}Responses to the question "Did you feel uncomfortable making the choice?"}
\end{table}


\textbf{Q5.} The participants choice of dog described in their responses to the question "Why did you pick the dog you picked?" are coded as being either a choice based on feelings or rationale (Table \ref{tab:rati}). Six participants (2, 3, 4, 5, 6, 7, 9) made a rational decision on their choice of dog. They base their decision on which dog they felt was best capable of fighting. In contrary participant 1 and 8 based their choice of dog on feelings. Participant 1 chose the husky because he thought it was cool but speculate that the boxer might have been more capable of fighting. Participant 8 did not choose the boxer because he was too emotionally attached to it and he wouldn't want to see it suffer. His response raises the question whether the other participants seemingly rationale choices are made on the basis of choosing the dog that would suffer the least. If that is the case then the participants choices are based on feelings in the sense that they choose the dog that will have the least emotional impact on them.

\begin{table}[h]
\centering
\begin{tabular}{l l l l}
\hline
\textbf{Rational or emotional}\\
\textbf{choice of dog?} & Rational & Emotional & Other \\
\hline
Male & 3 & 2 & 0 \\
Female & 4 & 0 & 0 \\
\textbf{Total} & 7 & 2 & 0 \\
\hline
\end{tabular}
\caption{\label{tab:rati}Expression of rational or emotional based choice in responses to the question "Why did you pick the dog you picked?"}
\end{table}


\textbf{Q6.} As can be seen in Table \ref{tab:like}, the participants liked their dogs. Only one participant (5) express that he did not feel a relation to the dog. However he also expresses that he chose a non-serious name to avoid getting too close to the dog. Interestingly two other participants (3, 9) also express a similar rationale for choosing a non-serious name for their dog. 

\begin{table}[h]
\centering
\begin{tabular}{l l l l}
\hline
\textbf{Did you like}\\
\textbf{your dog?} & Yes & No & Other \\
\hline
Male & 4 & 0 & 1 \\
Female & 4 & 0 & 0 \\
\textbf{Total} & 8 & 0 & 1 \\
\hline
\end{tabular}
\caption{\label{tab:like}Responses to the question "Did you like your dog?"}
\end{table}


\textbf{Q7.} The responses to the question "What would happen if you leashed your dog" can be divided into three categories. Leashing the dog would affect the relationship to the dog, the control over the dog or a believe that the action of leashing was a choice of right and wrong in terms of progressing in the game. The last notion has been coded as "other", see Table \ref{tab:leash}. Only two participants (6, 8) chose to leash the dog however both of them expressed that they thought it was the correct option, implying that they were afraid of the consequences in terms of progression in the game if they did not leash it. Participant 1, 2, 3, 7 and 9 express that they believe their relationship with the dog would be improved by not leashing it. Participant 4 and 5 both express that leashing it would have made the dog easier to control, but chose to not leash it. Which raises the question; what was their consideration behind not leashing it? Do they, in line with the majority of the participants, think that not leashing the dog can improve their relationship or do they too feel that the game proposes a right / wrong solution,

\begin{table}[h]
\centering
\begin{tabular}{l l l l}
\hline
\textbf{Believed consequence of}\\
\textbf{leashing dog?} & Relationship & Control & Other \\
\hline
Male & 3 & 1 & 1 \\
Female & 2 & 1 & 1 \\
\textbf{Total} & 5 & 2 & 2 \\
\hline
\end{tabular}
\caption{\label{tab:leash}Expression of believed consequences of leashing the dog in responses to the question "What would happen if you leashed your dog?"}
\end{table}


\textbf{Q8.} Only two participants (1, 2) felt responsible for the dogs their dog killed. It is worth noting that these were the two participants who participated in the first test session which may indicate a bias. Participant 4 and 7 are vague in their responses. Participant 7 states that she felt a bit responsible but that she would rather have that the other dog died than her own dog died. (Table \ref{tab:resp})

\begin{table}[h]
\centering
\begin{tabular}{l l l l}
\hline
\textbf{Did you feel responsible for}\\
\textbf{the dogs your dog killed?} & Yes & No & Other \\
\hline
Male & 1 & 4 & 0 \\
Female & 1 & 1 & 2 \\
\textbf{Total} & 2 & 5 & 2 \\
\hline
\end{tabular}
\caption{\label{tab:resp}Responses to the question "Did you feel responsible for the dogs your dog killed?"}
\end{table}


\textbf{Q9.} In general the participants express that they feel sadness when their dog died. As can be seen in Table \ref{tab:sad} three participants did not have a clear cut answer to the question. Participant 4 elaborates that she had prepared herself for it the moment she caught the dog. She knew that death was inevitable, implying that she was emotionally prepared for it to happen. Her response implies that she had to emotionally detach herself from the dog to not feel sad which in turn implies that she would have felt sad if she hadn't predicted its destiny. Participant 6 felt sad when the dog died not because of any emotional attachment to the dog but because she lost the game. Lastly participant 9 express that he did not feel sad, instead he felt anger towards the dog that killed his dog. Where the anger stems from is unclear and we can only speculate that either he was in fact emotionally attached to his dog or that he, in line with participant 6, was sad that he lost the game.

\begin{table}[h]
\centering
\begin{tabular}{l l l l}
\hline
\textbf{Did you feel sad when}\\
\textbf{your dog died?} & Yes & No & Other \\
\hline
Male & 3 & 1 & 1 \\
Female & 2 & 0 & 2 \\
\textbf{Total} & 5 & 1 & 3 \\
\hline
\end{tabular}
\caption{\label{tab:sad}Responses to the question "Did you feel sad when your dog died?"}
\end{table}


\textbf{Q10.} The majority of the participants felt responsible for the death of their dog (Table \ref{tab:death}). Only one participant (5) states that because he tried to run from the battle it was not his responsibility. Participant 4 states that she decided that she did not want to feel anything. Lastly participant 6 express that because she felt she was cheated she did not feel a great degree of responsibility.

\begin{table}[h]
\centering
\begin{tabular}{l l l l}
\hline
\textbf{Did you feel responsible for}\\
\textbf{the death of your dog?} & Yes & No & Other \\
\hline
Male & 4 & 1 & 0 \\
Female & 2 & 0 & 2 \\
\textbf{Total} & 6 & 1 & 2 \\
\hline
\end{tabular}
\caption{\label{tab:death}Responses to the question "Did you feel responsible for the death of your dog?"}
\end{table}


\textbf{Q11/Q12.} To understand the participants experience of perceived agency on the fights of the game they were asked two questions. "How does the battle system work?" and "Which options were most effective?". The participants responses to the questions has been coded as either the participant believed that hey had agency in the fights or they did not. The responses are illustrated in Table \ref{tab:agen}. Only one participant's (2) statement is unclear. She believes she can use items in the fights later in the game. However she also express that it seemed like her options had no effect. The majority of the participants believes they have agency in the fights though some of them are confused about the system. Only one participant (9) believed that he had no agency. He states that the realisation of his lack of agency came in the beginning of the second fight he was in. He states that it changed his perception of his role in the game, that he felt like he was becoming a spectator and that this notion made the message of the game stronger. His definition of the game's message is vague. He thinks the game is trying to convey that the protagonist is in a catch-22, which in turn creates awareness about poverty and what you have to do  to survive.

\begin{table}[h]
\centering
\begin{tabular}{l l l l}
\hline
\textbf{Perceived agency}\\
\textbf{in the fights?} & Agency & No agency & Other \\
\hline
Male & 4 & 1 & 0 \\
Female & 3 & 0 & 1 \\
\textbf{Total} & 7 & 1 & 1 \\
\hline
\end{tabular}
\caption{\label{tab:agen}Expression of perceived agency in the responses to the questions "How does the battle system work?" and "Which options were most effective?"}
\end{table}


\textbf{Q13.} To the question "Do you think it was an unethical game?", the majority of the participants answered no. Participant 6 answered that it did not matter as the game is satire (Table \ref{tab:ethi}). Only participant 5 and 8 answered yes. Participant 5 elaborates that he believes the Mexicans will oppose the game, but that in a sense the game is not more unethical than other games. He believes that human fights is more unethical, implying that the while unethical the game can exist under the socially accepted threshold of unethically in games. Participant 8 elaborates that the humor in the game makes it very unethical, but that its existence is still justified. Participants 1, 2, 3, 4 & 7 elaborates that the game makes you are aware and puts you in the horrible situation of the game, implying that your actions are justified by the situation you are in. Participant 4 and 7's only real complaint was that they wanted a disclaimer for the blood on the dogs.

\begin{table}[h]
\centering
\begin{tabular}{l l l l}
\hline
\textbf{Do you think it was}\\
\textbf{an unethical game?} & Yes & No & Other \\
\hline
Male & 2 & 3 & 0 \\
Female & 0 & 3 & 1 \\
\textbf{Total} & 2 & 6 & 1 \\
\hline
\end{tabular}
\caption{\label{tab:ethi}Responses to the question "Do you think it was an unethical game?"}
\end{table}


\textbf{Q14.} When asked if the game was moralising the majority of the participants answered no. Participant 4 answered yes and participant 7 was not sure (Table \ref{tab:mora}). Participants 1, 2, 3 and 5 elaborates that the game itself does not tell you what is right or wrong. Participant 4 states that the game is trying to hint something. She is not defining what the game is hinting but says the game is made to make you feel bad by dismembering and putting blood on cute dogs. Participant 7 also cannot define what the moral of the game is but that she feels it is tough that the game forces her to downprioritise her dog.

\begin{table}[h]
\centering
\begin{tabular}{l l l l}
\hline
\textbf{Is the game}\\
\textbf{moralizing?} & Yes & No & Other \\
\hline
Male & 0 & 5 & 0 \\
Female & 1 & 2 & 1 \\
\textbf{Total} & 1 & 7 & 1 \\
\hline
\end{tabular}
\caption{\label{tab:mora}Responses to the question "Is the game moralising?"}
\end{table}


\textbf{Q15.} To the question "What do you think about dog fighting?" all of the participants expressed disgust of a varying degree towards it. 


\textbf{Q16.} As can be seen in Table \ref{tab:thin} six participants express that the game has made them think more about dogfighting. However four of them express that they only do so temporarily while playing the game or only until short after having stopped playing. One participant (4) express that she does not think about dogfighting because there is nothing to do about it. Another participant (3) express that the game has made him think about the setting than the actual dogfights.

\begin{table}[h]
\centering
\begin{tabular}{l l l l}
\hline
\textbf{Has this game made you think}\\
\textbf{more about dogfighting?} & Yes & No & Other \\
\hline
Male & 4 & 0 & 1 \\
Female & 2 & 1 & 1 \\
\textbf{Total} & 6 & 1 & 2 \\
\hline
\end{tabular}
\caption{\label{tab:thin}Responses to the question "Has this game made you think more about dog fighting?"}
\end{table}


\textbf{Q17.} All of the participants express, to a varying degree, that they believe dogfighters love their dogs. Participant 8 furthermore express that he believe that the dogfight spectators do not love dogs. A belief participant 2 and 4 share.

\section{Discussion}
The test data is compiled from two test sessions at ITU. As already mentioned in chapter \ref(sessions) differences in the interview questions and implementation of the game has probably biased the test results. Before we discuss our test results in the perspective of our initial research questions \ref{ProbStat} we will present the bias of the test results and argue for its validity. \\

Despite the differences between the two test sessions and the bias it may have created we believe that the data is comparable across sessions as long as we look for and document the outlying differences between the data of the two test sessions. In fact comparing the sessions can help us get indications of what impact the differences in the implementation of the game has had. The differences are hard to point out but two interesting things stands out. The first is that only the two, session 1, participants (1, 2) felt responsible for the dogs their dog killed. The second being that the depiction of the man to which you offer sexual favours apparently has had an effect on the participants choice of destiny. \\

As mentioned in the paragraph of "Q4" in chapter \ref{interview}, participant 1 spend one minute deciding her destiny (sexual favours versus dog fighting). Participant 2 on the other hand chose sexual favours in a matter of seconds. The participants (1, 2) behaviour is unlike the behaviour of the majority of the test participants in session 2. It seems like the session 1 participants were more in doubt of which of the options were most ethically sound. The reason behind this may be found in the fact that the man to which you could offer sexual favours only were depicted in the second implementation of the game \citep[see]{gringo}. Participant 7 and 9 implied that their choice of destiny partly depended on the looks of the man, that he was too repulsive. Participant 8 state that he felt the game wanted him to choose dog fight. We can only speculate what made him feel like that but the depiction of the man as being very repulsive may be the answer. \\

Why only the session 1 participants felt responsible for the dogs their dog killed may stem back to the problem mentioned above. If the choice of destiny was harder to justify for the participants (1, 2) they may have carried that sense of doubt with them into the fights questioning whether dog fighting was the moral choice as their dog kills the other dog. \\

The results in chapter \ref{interview} was presented divided by gender. However there does not seem to be a lot of interesting gender differences in the data. The one that stands out the most is the choice of dogs. The majority of the participants chose the boxer on the basis that it was best fit for dog fighting. However among the male participants the husky was the favourite and no females picked the husky. It seems like the male participants were more willing to run the risk of choosing a more 'cool looking' dog over the safe bet of the boxer. This points us in the direction that the fear of seeing the dog get hurt or killed in a fight were a deciding factor for the female participants.  \\


\begin{figure} 
	\centering
    \includegraphics[width=0.5\textwidth]{gringo.png}
    \caption{The man to which you can offer sexual favours in the game}
    \label{fig:gringo}
\end{figure}


\textbf{RQ1: Can we create unethical affordances, namely dog fighting, in a game without making it an unethical game?}\

The participants responses to Q13 and Q14 indicate that our game is neither perceived as being unethical nor moralising. The trend in the data shows us that the participants bought the premise of the game and felt that the situation you - the protagonist - is placed in justifies your otherwise unethical actions. Unfortunately their responses does not answer our research question in the general sense the question is asked. We know how test participants react to the game after having played it, but we do not have any data on how the common consumer will react to a game like this once it surfaces in public. The game contains enough sensitive subjects to create a solid foundation for public outcry. Even before its release in 2009 \textit{Resident Evil 5} \citep[RE5]{game:re} med massive criticism for its portrayal and dealing with African culture and Africans as zombies \citep{harrer2015black}. Another example from 2015 is when Serious Games had to remove a mini-game, popularly known as "Slave Tetris", in their educational serious game \textit{Playing History 2: Slave Trade}  due to a public outcry \citep{kotaku}. Common to both games is that people defending them claim that the games are subject to misinterpretation by people who have not played them \citep{harrer2015black, kotaku, mtv}. Without going further into the arguments for and against the games it seems like games containing controversial subject matter are subject to superficial interpretations of the games purposes. A hypothesis supported by Deterding's \citeyear{deterding2016mechanic} work showing that the framing of the game plays an important part in the reception of the game.

\blockquote{...Playing History 2 travelled public discourse in the form of a single screenshot of the ?Slave Tetris? level, designed in a child-friendly cartoon look, cutting away the internal critique of the modeled proceedings later in the game. By framing itself as a game for children, it activated a children?s entertainment game frame, which clashed with the serious subject in the audience?s perception.}\citep{deterding2016mechanic}\

Whether our framing of \textit{Rat Kid Dog Fight!} publicly justifies its existence we cannot know. One participant (5) expressed that he believed the Mexicans would oppose the game. The implementation of the game is very much a prototype and its fair to say that the depiction of Mexico City's slum culture is based on loose evidence and gut feelings. The above example of games that fail to frame their purpose should stand as a reminder that we have to carefully analyse and design for the right framing of the game in a future implementation.


\textbf{RQ2:  Will the player be able to feel empathy for the dogs, while being reliant on them fighting?}
The unanimous responses to Q17 ("Do dog fighters love their dogs?") illustrates well that the players had or could imagine having an emotional relation to their fighting dog. However it is hard to find a clear cut answer to whether our test participants felt empathy for their dogs in the game. To try and deduct an answer from the test results we have to broaden our investigation and look for trends that reveal the participants emotional engagement with the game starting with the participants relation to the protagonist of the game. In the paragraph of "Chosen protagonist name" in chapter \ref{ingamedec} we raise the question whether the choice of name for the protagonist is a reflection of the participants emotional engagement with the game. \

One of the participants (6) who chose a non-serious name was experiencing the whole game as a satire, she clearly had a humorous distance to the entire play experience. Another participant (9) who chose a non-serious name states that he did not identify with the protagonist and that it was a conscious decision to give the protagonist a non-serious name. However data from three of the participants (1, 4, 7) who chose a serious name indicates that they did not identify with the protagonist. The data does not provide a conclusive answer but when correlated with the data of the questions involving the participants dogs it seems like the choice of name for both dog and protagonist is a reflection of how willing the participant is to let herself be emotionally engaged in the game. 

\In contradiction to the chosen protagonist name only two participants chose to give their dog a serious name. As described in the paragraph of "Q6" in chapter \ref{interview} three participants expressed that they chose a non-serious name for their dog to avoid getting too close to it emotionally. This points us in the direction that the participants use the naming of both the dog and the protagonist to adjust their emotional engagement to a level they are comfortable with. We interpret the shift in serious names from 5 (protagonist) to 2 (dog) as a sign of raised emotional engagement over the course of the game. The players choose to give the dogs non-serious names not because it is an expression of their relation to the dog itself but a way of adjusting their emotional engagement so they are less vulnerable for whatever they would meet later in the game. This correlates well with the data of both Q6 and Q7 in chapter \ref{interview}. 

\The participants liked their dogs and they chose not to leash them to avoid damaging their relationship to them. Furthermore the majority of the participants expressed sadness upon the dogs death and a sense of responsibility towards the dog. The discussion presented in this paragraph show that the players try to create a distance to the dogs and the game via non-serious naming of their dog. Upon naming the dog the participants were aware that the dog was going to be used in for dog fighting. Even so they still made decisions that signifies emotional bonding to the dog. Hence we believe the majority of the participants felt empathy for their dog. 



\textbf{RQ3: How does the player feel, when they lose their dogs, while lacking the agency to prevent it?}
This research question can hardly be answered based on the data acquired. Only one participant (9) realised that he did not have any agency in the fights. He expressed his experience as changing from being an actor to becoming a spectator to the fights this experience he felt conveyed the message of the game in a stronger sense, though he lacked a proper definition of the message of the game. It seems that the realisation of the lack of agency needs more time to mature in the player. Some of the participants were confused about the fighting UI and some speculated that it might not have made a difference what they chose but still only one came to the full realisation. A couple of the participants even believed that some of the attacks were better than the other. To test our hypothesis, that the realisation of lack of agency results in a deeper perception of realism, we have to prolong the play session to include more fights or change the UI to make it easier to realise the lack of agency. That the design of the fighting UI is good at hiding the lack of agency correlates well with the notion of \citet{wardrip209agency} that the restriction of interaction can prevent the breakdown of the illusion of agency. We chose to use a UI that resembles that of the fights in the \textit{Pok�mon} gameboy games. A simple fight system UI restricted by being turn-based and having only a few options for actions. In the fights of our game we wanted to create the illusion that all the possible actions, if not possible now, will be possible later. For instance upon choosing to use an item the player will be told that they currently are not possessing any items implying that once you have collected an item you can use it. These illusion seems to work however we had thought that the participants would have realised the lack of agency sooner than they did.


What can we deduct from participant 9's experience of lack of agency if we suppose that his response would be the general response from players of the game. Upon realisation he knew that neither of his actions made sense other than pacing the speed of the fights. In another response he expressed that the game became boring towards the end even though he felt that the realisation made the purpose of the game shine through. For a future implementation of this game we have to consider how to weigh game purpose (partially conveyed through the realisation of lack of agency) against playability and continuation desire. 






IS DOG FIGHTING THE BETTER ALTERNATIVE?
One of the things we were curious about is whether dog fighting is perceived as a better alternative to prostitution. It is no secret that in our - the designers - mind dog fighting is a better alternative. Our test data indicate that the majority of the participants felt the same way but we cannot ignore the data that resists this notion. A trend in the data reveals that the visual appearance of the man you are to prostitute yourself to has had an impact on the choice. We might also have to question whether the choice is affected by the player's cultural background.

One participant(2) spend a minute making the decision. She was clearly emotionally affected by the game and the choice made her feel uncomfortable. Interestingly she was part of the first test session. In the implementation of the game used for the first session the man to which you can offer sexual favours is not depicted. In the second implementation of the game he is depicted as being very repulsive.  \commenting{Insert picture of man} The fact that he is not depicted in the first test session may have made the choice less unambiguous. This thought is supported by two other participants(7, 9) statements to the question "Why did you pick dog fighting over sexual favours?". They state that the man's uglyness played a role in their decision to pick dog fighting. A fourth participant(8) states that he felt the game wanted him to choose dog fighting over sexual favours. Whether the depiction of the man has had an impact on this can only be speculated but it seems like the new implementation of the game pushes the player more towards picking dog fighting over sexual favours.

Two(1, 6) of the three participants who chose sexual favours over dog fighting stated that they chose sexual favours out of curiosity. Observations of the participants, a male and a female, indicate that they were taking the seriousness of the game lightly and were having fun with it. They seemed to have a satirical distance to the game. The male participant(1) states that he chose the sexual favours because it was more extreme. The female(2) states that her choice may have been biased because she was sitting at a test. Their data indicate that they felt that dog fighting was the moral choice but that they chose sexual favours just to see what would happen in the game. Interestingly the last participant(4) chose sexual favours because she felt that it was the moral choice. Born and raised in Hong-Kong her cultural background is different from the rest of the participants (native Danes). Her reasoning being that no one dies in the sexual favours scenario. She is aware of, and believes that her opinion and choice is different from the other test participants. However she herself states that it is because of her cultural background.


\section{Future Game Development}
\label{Future}
In this section we will discuss under which conditions the prototype would be best evolved into an actual game.

\subsection{Game Mechanics}

More of a real game could be done

Setting seemed to work very well. \commenting{ethics book reference} Could be shortened or shown more cinematically.

the selection of sex or dog fight could be done by the character, as in many games (ref gta), as to not make a street kid simulator.

As mentioned in \ref{testResults} none of our testers understood, that there was no agency in the battles, while they did comment on a lack of understanding of how the selectable options worked. It should therefore be possible to make combat actually have agency, without the game becoming too unethical. Still, we believe the gameplay would be better served without any agency in the battle, and the ideal solution would then be one, where the player realized that there is not any agency.

shouts instead of pokemon interface - not cheating, but focus on and expand where the player actually has agency.


Focus on where the actual gameplay is. Training, breeding catching and random events.




Less humour and pokémon references could make it a more ethical game.

\subsection{Audience and Platform}

As mentioned in \ref{testResults}, one of our testers even suggested collaborating with NGO's, to create awareness around the conditions of street kids and illegal dog fighting. 

Both art exibition, schools and retail/steam, could be possible. Interestingly, we believe that the latter would need more attention to the ethical considerations, possibly even downgrading the 
A game where the intended purpose is either educational or artistic, would also be easier to find funding for. 
Hence, we find that an initial release for either schools or museums, with the possibility of a latter release on Steam \commenting{insert reference} would be ideal.

One problem with collaborating with schools or NGO's, is that the players might find the game more moralizing, even if would not in another context, simply because of these associations.

 Important that the features reflect the audience and agenda.



\section{Conclusion}
Ethical or not?\

Moral or not? General reaction as ethical and not moralizing. Emil thought humour made it unethical. Thora thought of it as a piece of art and hence morality and ethics were not important, just the message.\

Reaction to no agency. \

Dilemma of research - blind testing vs. questions before testing.
	how would we do the test over

Implementation and platform for the future game.\

While sensibilities might change as the norms of our society does, it is always the responsibility of the artist to push at and test the boundaries of those sensibilities, as well as their legitimacy. \

Woof!\

%% References
%%
%% Following citation commands can be used in the body text:
%% Usage of \cite is as follows:
%%   \citep{key}          ==>>  [#]
%%   \cite[chap. 2]{key} ==>>  [#, chap. 2]
%%   \citet{key}         ==>>  Author [#]

%% References with bibTeX database:

\section*{\refname}
\bibliographystyle{elsarticle-harv}
\bibliography{references}

\


%% Authors are advised to submit their bibtex database files. They are
%% requested to list a bibtex style file in the manuscript if they do
%% not want to use model1-num-names.bst.

%% References without bibTeX database:

% \begin{thebibliography}{00}

%% \bibitem must have the following form:
%%   \bibitem{key}...
%%

% \bibitem{}

% \end{thebibliography}

%% The Appendices part is started with the command \appendix;
%% appendix sections are then done as normal sections
\appendix
\renewcommand*{\thesection}{\Alph{section}}


\section{Appendix}
\label{appendix}

\end{document}

%%
%% End of file `elsarticle-template-1-num.tex'.